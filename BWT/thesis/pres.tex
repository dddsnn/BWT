\documentclass{beamer}

\usepackage{tabularx}
\usepackage{color}
\usepackage{multirow}
\usepackage{booktabs}
\usepackage{xparse}
\usepackage{array}
\usepackage{pifont}
\usepackage{graphicx}
\usepackage{tabularx}

\beamertemplatenavigationsymbolsempty
\newcolumntype{t}{>{\fontfamily{\ttdefault}\selectfont}c}
\newcolumntype{y}{>{\fontfamily{\ttdefault}\selectfont}r}
\newcolumntype{u}{>{\fontfamily{\ttdefault}\selectfont}l}
\setbeamertemplate{footline}[frame number]
\NewDocumentCommand{\rot}{O{25} O{1em} m}{\makebox[#2][l]{\rotatebox{#1}{#3}}}

\makeatletter
{\obeylines\gdef\bt@eol{^^M}}
\newenvironment{breakabletexttt}
  {\ttfamily\hfuzz=0.4em
   \list{}{\leftmargin=2em
           \itemindent=-\leftmargin
           \listparindent=-\leftmargin
           \parsep=0pt}
   \item\relax\obeylines\breakable@texttt}
  {\endlist}
\def\breakable@texttt#1{%
  \ifx#1\end
  \expandafter\end
  \else
    \expandafter\ifx\bt@eol#1%
      #1%
    \else
      \string#1\hskip1sp
    \fi
    \expandafter\breakable@texttt
  \fi}
\makeatother

\title{Burrows-Wheeler compression with modified sort orders and exceptions to
the MTF phase, and their impact on the compression rate}
\author{Marc Lehmann}
\date{}

\begin{document}

\frame{\titlepage}

\begin{frame}
\frametitle{Burrows-Wheeler compression}

\begin{itemize}
  \item M. Burrows and D.J. Wheeler in 1994
  \item lossless compression
  \item context-based
  \item 3 stages in basic form
\end{itemize}

\pause

Definitions:

\begin{description}
  \item[symbol] smallest logical unit of information
  \item[string] sequence of symbols
\end{description}

\end{frame}

\begin{frame}
\frametitle{The Burrows-Wheeler Transform}

Example: \texttt{mississippi}

\begin{itemize}
\item Write all cyclic shifts into a table.
\end{itemize}

\begin{table}
\centering
\begin{tabular}{|r||ttttttttttt|}
\hline
0 & m & i & s & s & i & s & s & i & p & p & i \\
1 & i & s & s & i & s & s & i & p & p & i & m \\
2 & s & s & i & s & s & i & p & p & i & m & i \\
3 & s & i & s & s & i & p & p & i & m & i & s \\
4 & i & s & s & i & p & p & i & m & i & s & s \\
5 & s & s & i & p & p & i & m & i & s & s & i \\
6 & s & i & p & p & i & m & i & s & s & i & s \\
7 & i & p & p & i & m & i & s & s & i & s & s \\
8 & p & p & i & m & i & s & s & i & s & s & i \\
9 & p & i & m & i & s & s & i & s & s & i & p \\
10 & i & m & i & s & s & i & s & s & i & p & p \\
\hline
\end{tabular}
\end{table}

\end{frame}

\begin{frame}
\frametitle{The BWT}

\begin{itemize}
  \item Sort the table lexicographically to get the \emph{BW table}.
  \item Last column is the output of the transform (\emph{BW code}).
\end{itemize}

\begin{table}
\centering
\begin{tabular}{|r||tttttttttt|t|}
\hline
0 & i & m & i & s & s & i & s & s & i & p & p \\
1 & i & p & p & i & m & i & s & s & i & s & s \\
2 & i & s & s & i & p & p & i & m & i & s & s \\
3 & i & s & s & i & s & s & i & p & p & i & m \\
4 & m & i & s & s & i & s & s & i & p & p & i \\
5 & p & i & m & i & s & s & i & s & s & i & p \\
6 & p & p & i & m & i & s & s & i & s & s & i \\
7 & s & i & p & p & i & m & i & s & s & i & s \\
8 & s & i & s & s & i & p & p & i & m & i & s \\
9 & s & s & i & p & p & i & m & i & s & s & i \\
10 & s & s & i & s & s & i & p & p & i & m & i \\
\hline
\end{tabular}
\end{table}

\end{frame}

\begin{frame}
\frametitle{The BWT}
\framesubtitle{Context Blocks}

\begin{itemize}
  \item \emph{Context block}: block of BW code corresponding to rows with a
  specific symbol at the beginning.
\end{itemize}

\begin{table}
\centering
\begin{tabular}{|r||tttttttttt|t|}
\hline
0 & \color{red}{i} & m & i & s & s & i & s & s & i & p & \color{red}{p} \\
1 & \color{red}{i} & p & p & i & m & i & s & s & i & s & \color{red}{s} \\
2 & \color{red}{i} & s & s & i & p & p & i & m & i & s & \color{red}{s} \\
3 & \color{red}{i} & s & s & i & s & s & i & p & p & i & \color{red}{m} \\
4 & \color{blue}{m} & i & s & s & i & s & s & i & p & p &
\color{blue}{i} \\
5 & \color{green}{p} & i & m & i & s & s & i & s & s & i & \color{green}{p} \\
6 & \color{green}{p} & p & i & m & i & s & s & i & s & s & \color{green}{i} \\
7 & \color{cyan}{s} & i & p & p & i & m & i & s & s & i & \color{cyan}{s} \\
8 & \color{cyan}{s} & i & s & s & i & p & p & i & m & i & \color{cyan}{s} \\
9 & \color{cyan}{s} & s & i & p & p & i & m & i & s & s & \color{cyan}{i} \\
10 & \color{cyan}{s} & s & i & s & s & i & p & p & i & m & \color{cyan}{i} \\
\hline
\end{tabular}
\end{table}

\end{frame}

\begin{frame}
\frametitle{The BWT}
\framesubtitle{Reversibility}

\begin{itemize}
  \item Every column is a permutation of the input
  \item First column can be reconstructed by sorting the last
\end{itemize}

\begin{table}
\centering
\begin{tabular}{|r||tttttttttt|t|l}
\cline{1-12}
0 & \color{red}{i} & \colorbox{red}{m} & i & s & s & i & s & s & i & p & p & \\
1 & \color{blue}{i} & \colorbox{blue}{p} & \colorbox{blue}{p} & i & m & i & s &
s & i & s & s & \\
2 & \color{green}{i} & \colorbox{green}{s} & \colorbox{green}{s} &
\colorbox{green}{i} & \colorbox{green}{p} & p & i & m & i & s & s & \\
3 & \color{cyan}{i} & \colorbox{cyan}{s} & \colorbox{cyan}{s} &
\colorbox{cyan}{i} & \colorbox{cyan}{s} & s & i & p & p & i & m & \\
4 & \colorbox{red}{m} & i & s & s & i & s & s & i & p & p & \color{red}{i} & 0
\\
5 & p & i & m & i & s & s & i & s & s & i & p & \\
6 & \colorbox{blue}{p} & \colorbox{blue}{p} & i & m & i & s & s & i & s & s &
\color{blue}{i} & 1 \\
7 & s & i & p & p & i & m & i & s & s & i & s & \\
8 & s & i & s & s & i & p & p & i & m & i & s & \\
9 & \colorbox{green}{s} & \colorbox{green}{s} & \colorbox{green}{i} &
\colorbox{green}{p} & p & i & m & i & s & s & \color{green}{i} & 2 \\
10 & \colorbox{cyan}{s} & \colorbox{cyan}{s} & \colorbox{cyan}{i} &
\colorbox{cyan}{s} & s & i & p & p & i & m & \color{cyan}{i} & 3 \\
\cline{1-12}
\end{tabular}
\end{table}

\end{frame}

\begin{frame}
\frametitle{The BWT}
\framesubtitle{Reversibility}

\begin{itemize}
  \item Start index is also needed
  \item \(i\)-th occurrence in the first column corresponds to \(i\)-th
  occurrence of the last column
\end{itemize}

\begin{table}
\centering
\begin{tabular}{|r||tttttttttt|t|l}
\cline{1-12}
0 & \color{red}{i} & \colorbox{red}{m} & i & s & s & i & s & s & i & p & p & \\
1 & \color{blue}{i} & \colorbox{blue}{p} & \colorbox{blue}{p} & i & m & i & s &
s & i & s & s & \\
2 & \color{green}{i} & \colorbox{green}{s} & \colorbox{green}{s} &
\colorbox{green}{i} & \colorbox{green}{p} & p & i & m & i & s & s & \\
3 & \color{cyan}{i} & \colorbox{cyan}{s} & \colorbox{cyan}{s} &
\colorbox{cyan}{i} & \colorbox{cyan}{s} & s & i & p & p & i & m & \\
4 & \colorbox{red}{m} & i & s & s & i & s & s & i & p & p & \color{red}{i} & 0
\\
5 & p & i & m & i & s & s & i & s & s & i & p & \\
6 & \colorbox{blue}{p} & \colorbox{blue}{p} & i & m & i & s & s & i & s & s &
\color{blue}{i} & 1 \\
7 & s & i & p & p & i & m & i & s & s & i & s & \\
8 & s & i & s & s & i & p & p & i & m & i & s & \\
9 & \colorbox{green}{s} & \colorbox{green}{s} & \colorbox{green}{i} &
\colorbox{green}{p} & p & i & m & i & s & s & \color{green}{i} & 2 \\
10 & \colorbox{cyan}{s} & \colorbox{cyan}{s} & \colorbox{cyan}{i} &
\colorbox{cyan}{s} & s & i & p & p & i & m & \color{cyan}{i} & 3 \\
\cline{1-12}
\end{tabular}
\end{table}

\end{frame}

\begin{frame}
\frametitle{The BWT}
\framesubtitle{The Effect}

\begin{itemize}
  \item Substrings of the input beginning with the same symbols are sorted one
  below the other
  \item BW code are the symbols immediately preceding them
  \item Usually only a few distinct symbols in a context block, many runs of the
  same symbol
\end{itemize}

\pause

For example, context block corresponding to ``\texttt{nd }'' in book1 (first
100 symbols):
\begin{breakabletexttt}
eaeaaAaaiaaaaaaaaauaaaaoaaaaiaaaiiaauaauaauaiaaaaiuaaaaaaaaaaaaaaaaaaaaaiaaaaaaAaaaaaaaaaaaaaaaaaaaa
\end{breakabletexttt}

%"nd ": eaeaaAaaiaaaaaaaaauaaaaoaaaaiaaaiiaauaauaauaiaaaaiuaaaaaaaaaaaaaaaaaaaaaiaaaaaaAaaaaaaaaaaaaaaaaaaaaaaiaaaaaaaaaAaaaaaaaaaaaaaaaaaaaAaaaaAaaaaaaaaaaaaaaaAaaaaaAaaaaaaaaaaaaaaaaaAaaaaaaaaaaaaaaaaaaaaaaaAaAaaAaaaaaAaaAAaaaaaAaaAaaaaaaaaaaaaAAaaAaaaaaaaaaaaaaaaaaauaaaaaaaaaaaaiaaaaaaaaaaaaaaaaauaaaaiaaauaaaaaaaaaoaaaaaaAaaAuaaaaoaeaeauaaaaaaaaaiiuaaaaauaaaaaaaeaaaaaaAaaaaaAAaaaeauuaiaiauauauiaaaauauuaaaaaoaaiaiaaaaaaaaaaaaaaaaaiaaaaaaaeiaAaaiaaaaaaaiouuauaaiaaaaaaaaaaeaaaaaaaaaaaaaaaaaauaaaauaaaaaaaaauoaaiaaoaoaaaaaaaaaoaiaaaaaaaaaaaaaiaaaaaaaaaaaaAaaauaaaaauaaaaaauiauaauauoiaoauiaaaaAaaaaaaaaaaauaaaaaaaaaaaaiaaaaaaaaauauaaaaaAaaAauiaaAaaaAauauaaaaaaaaaaaaauaaauaiaAaaaaaAAaaeaiaaaaiaaaauuuaaaaaaaaaaaaaaaaaaaaaoaaaaaaaaaaaaaaaaaaaaaauAaaaaaaaaaiaaaaaaaaaaaaiaaaaaaaaaiaaaauaaaaaaaaaaaaaaaaaaaaaaaaaaaiaaaaaaaaaaaaaaaaaaaaaaeaaaaaaauaaaaaaaaaaAaaaaaaaaaaaaiaaiaaaaaaaaaaaaaaaaaaaaaaaaaaaaaaaaaaaaaaaaaaaaaaaauaaaaaaaiaaaaaaaoaaaaaaaaaaaaaaaaaaaaaaaaaaaaaaaaaaiaaaaaaaaaaeaaaaaaaaaiaaaaaaaaaaaaaiaaiiaaaaaaaaaaaaaaaaaaaaaaaaaaaaaaaaaaaaaaaaaaaAaAaaaaaaaaaaaaaaaaaaaaaaaaaaaaaaaaaaaaaaAaaaiaiaaaaaaaAaAaaaaoaaaaaaaaaaaaaaaaaaaaaaaaaaaaaaaaaaaaaaaaaaaaaaaaaaaaaaaaaaaaauaaaaaaaaaaoiaaaaaaaaaaaaaaaaaaaaaaAauaauaaaaaiaaaaaaaaaaaaaauaaaaaaaaaaaaaaaaaaaaaaaaaaaaaaaaaaaaaaaaaaaaaaaaaaaaaiaaaaaaaaaaaaaaaaaaaaiauaaaaieeaauuueuaaaauaaaaaaaaaaaaaaaaaaaaaaaaaiaaaaAaaaaaaaaaaaaaaaaaaaaaaaaaaaAaaaaaaaaaaaaaaaaaaaaaaaaaaaaaiaaaaaaAaaaaaaaaaaaaaaaaaaaaaaaaauaaaaaaaaeaaaaaaaaaaaaaaaaaaaaaaaaaaauaaaaaaaaaaAAaaaaaaaaaAaaaaaAaaaAaaaaaaaaaaaaaaaaauaaaaaaaaaaaaAaaaaauaaaaaaaaaaaAaaaaaauaaaaaaaaaaaaaaaaaaaaaaaauauuuaaaaiaaaauaaaaauaaaaauaaaaiaiiAauaaiiaiiaaaaaaAaauiaaaaauauiiuiiaeuuiuuuauuuaaaiuaaaaaaaaaaaaaauaaaeeaaaaeaaouaaaaAaauiaiaaaauaaaaaAaaaaaaaaaaAaaaaAiaaAAaaAAaaaaaaaAAaaaaaaaaaAauaaAaAaaaaaaaaaaaauauaiauaaaauaaaauaiaaaaaaaaaaauuaaaaaaaaaaaaaaauuaaaaauaoaauaaaaaaaaaaaaaaaauaaaaauaiaaaaaaaaaaaaaaaaaAaaiaaaiaaaaiiaiaiaaauuAaaAaaaaaAaAaiuaaAaaeeeuaaaaaaaaaaaaaaaauioiiiiaaaaaaaaaaaaauaaaaaaaaaauuaaaaaaaaaaaaaaaaaaaaaaaAaaaaaaaaaaaaAiaaaaaaaaaaaaaaaaaaaaaaaaaaaaaaaaaaAAaaaaaaeaaaauaaaaaaaaaaaaaaaaaaaaaaaAaaaaaaaaaaaaaaaaaaaaaaaaaaaaaaaaaaaaaaaaaaaaaaaaaaaaaaaaaaaaaaaaaoaaaaaaaaaoaaiaaaaaaaaaaaaaaaaaauaiaiiooaaaaaaoaaaaaaaaaaAaaaaAuaaaaaaaaaiaaaaaiaaaaaaaaaaaaaaaaaeaaaaaaaAaaaaaaiaaaaaaaAaooaaaoaaaaaaaaaaaaaaaaaaaaaaaaaaaaaaaaaaaaaaaaaaaaaaaaaAaaaaaaaaaaaaaaaaaaAaaaaaaaaaaaoaaauaAAAaaaaaAAaaaAaaaAAAAAaAuuuaaaaaiaaaaaiaeaiuueaeeeaeieueuaieuueuoeeeaaeeuiieioaeaauuaueeeuuieaeeeaeeeeaeeeeeeeeuueeuuoeeeuuiaoaeaaaaaaauaAaaauauoaiaaaaaauaaaaaaaaaaaaaaaaaaiiaaaaaaaaaaaaaaaaaaAaaaaaiiiuiuiaaaaaaaaaaaaaiaaaaaaaaaaaaaaaaaaaaaoaaaaaaoaaaaAaaaaaaaaaaaaaaaaaaaaaaaaaaaaoaaaaaaaaaaaaaaiaaaaaaaAaaaaaaaaaaaaaaaaaaaaaaaaaaaaaaaaaaaaaaaaaAaaaaaaaiaaiaaaaaaaaaaaaauaaaaaaaaaaaaaaaaaAaaaaaaaaaaaaaaaaauaaaaaaaaaaaaaaaaaaaaaaoaaAaaaaaaaaaaaaaaaaaaaaaaaaaaaaaaaaaaaaaaaaaaaaaaaaaaaaiaaaaaaaaaaaaoaaaaaaaaaaaaaaaaaaaaaaaaaaaaaaaaaiaaaaaaaaaaaaaaaaAaaaaaaaaaaaaaaaaaaaaaAaaaaaaeaaaaaaaaaaaaaaaaaAaaaaaaaaaaaaaaaaaaaaaaaaaaAaaaaiaaaaaaaaaaaaaaaaaauaaaaaaaaaaaaaaaaaaaaaaaaaaaaaaaaaaaaaaaaiaaaaaAAaAAAaaaaaaaaaaaAaaAaaaAaaaaaAAaaaaaaaaaaauuaaiaaaaaaaaaaaAAaaaAaaaaaaaaaaaaaiaiaaaaaaiaaaaaaaaaaaaaaaaaaaaaaaaaaaaaaaaaaaaaaaaaaaaaaaiaaaAaiaa-aaaaaaaaaaaaaaaaaaaaaaaaaaaiaaaaaaaaaaaaaaaiaaaaaauaoaiaiaaauaaauiaauaaaeaaauaiuaaaiaaaAaauuauiuaaeaaaaaaiaaaauuAiiaaaaaaaiaaaaaeaaaaaaaaaaaaaauuaaaaaaaaeaaaaAaaaaaaaoauaaaiaaaaaauuaaaaaaaiaauaauaaaaaaaaaaaauaaaaaaaaoaaauaauaoaaauaaaaaaiaauoaaaaoaaaaAaaauauiaauuaAaaauaaaoaaaaaaaaaaaaaaaaaaaiiaiaaaaaaaaaaaoaaaaaaaaaaaAaaaAaaiiaaaaaaaaiAaaoaaaaaaAauaaaaaaaaaaaaaaaaaaaaaaaaaaiaaaaaaaaaaioaaaaaaaaaaaaaaaaiiaaaaaoaaaaaaaaiaaaaaaaaaaaaaiaaiauaauaaaaaAaaauaaeaaaauiiaaaauaauaiuiiiaaaaaaAaaAAaaaaaaaaaaaaaaaaaaaaaaaaaaAAauaaAAaaaaaaAaAaaaaaaaaaaaaaaaAaaaaaaaaaaaaaaAuaaaaaaaaAAaaaaaaaiaaiaiaaAaaaaAaaaaaaaaaaAAaaaaaaaaaaiaaaaaaaaaaiaiaaAaaeaoaiaaiiiiaaaaaaaaaaaaaaaaaaaoaaaaaaaaaaaaaaaaaaaaaaaaaAaaaaaauueaooaooaaaaaaaeaueuuuauaeiaaiuaauaeaiuaaiaiiaiAuiiiaaiaeaaiaoauAiaaaAaueuiuuueeuuaaiaaaaaaaaaaaiaaaaaaaaaaaaaaaaoaaaaaaaaaaaaaaaaoaaaaaaaaaaaaaaaaaauaaaaaaaaaaaaaaaaeaaaaaaaaauaaaaaaaaaaaaaaaaiiaiuuaaueuaiauaeaaaaoaaaaaaaaaaaaaaaaaaaaaaaiaaaaaaaaaaaaaaaaaaaaaaaiuaaaiaaaaauuaiaaaaaaAiaaaaeaaauuaaaaiaaAiuiuaaaaaaaaaaaaaaaaaaaaaaAaaAaaaaaaaaaaaaaaaaaaaaaaaaaaaaaaaaaaaaaaaaauiaaaAaaaaoAAaAaaAaaaAaaaaaaAaiaaaaAAAaAaaaAaaeieaaaiaiiiaAAuiaaaaaaaaaaaaaiaaaaaAaaaaaaaAaaaaaaaaaaaaaaaaaaaaaaaaaaaaauaaaaiuaaauaaaaiAiaaaaaaaoaaaaaauaaaaaaaaaaaaaaaaaaaAiaaaaaaaiaaaAAaaaaaaaAaaaAAaaaaaaaAAAAaaaAaiAiaaaaaiAAaaAaAAaaaaaauauaaaAAAAaauauaaaaaaaaiai
%". ":
% leyyytetylnyyykrnytehnnyadyyyrkesyyydalsyyyddlednyyydtrgkryesendydnekyayswnregsyrmdeycs.nntyhkdegyd!seeenyktntlenrrhydmyrwyensnypndbeyretmrnysetyetykyyyneymynpnhdengblseydttdnpsnn)yeyypdehkweylnellyytrefdnsnfeykssdklnyesglsewykekydtyyeylyrdnrdydslspaeyddyednmdhsrnyyddsyrefe)ysnegytre.mnnyyskyklbddsrshssetnr)ryrktenanlhddelwtepyyyemdawsdeeyylyyfsegteldtehlbtwy!!drt!ygnrtyunhl!!!!!!!t!egayedtykny!depglgtrkyywypdynd!!!!!!!!!!!!kweynywyfdytnengykenadnnmeyeeeeldmdnyndrydydndndtyrltlleyynryekdgesmeesoehyeeknnedtdedynynytnatlrtmddyk!kdd .nhetyemt!!c e ynsyrrytehlteerreengdmrrddsssytttysstadleeykysprlsetipohtfedserkdntggesmeewrrsrnenydnlewegmessstsllseng.y.rrttttdeeytsdmgr)rdtreytosneenledswIlghn.smeleegeostrslnesmdnehtshnmottwslgnnnyydeheesntnsulseneonyrnesesnldwrdsndrdtddrmttterdtghenmemdpeswe.eederrtdrtmdeedtenrnmeetggyddsltrpndynsyrtedesnerdtdrsnene.ewgersssneenefeteteherreedwtkswhlnnrdrdrrerrrrrrsedsarererydeyererrremrtrrdelrmrnsresrrrrrrrrrllrrmrrrsrrrrrrrrrydsettedehnergde.yeegtnsytmoottnoyey'tnnelsussnnyosdnttmyedyemtmnr.srsslstnsnndwtetrdewsesdeysenndenlefsedsgetrwstyeueetdkdresrlwdssoenrmnrrrrnrdsdssgmestsrhrrsserssgusnnlkmsnyfgrn!dyegtuyrrnnrytdstrneedk!teuwtos!essewkBBnnwkeetsherthegIsyteeyyllteeuasstmsyenyneermnyeessnhlyerhstnwwererrnrrtwretsseyyrsvnaeyentneeensytee!yegsgdtswdeehkerenmryenfsmtenoeekosroreeoloyaTlnedmyysntotgeysnkerfdemsaeeeeylkyyeemrgsnerenteshneessdynatnseddynssnrdpyktwkyydltslsettnydygnteyeuntdtnsnunndneeeeydttofkdwedmdpdsyededydnernrdmmytyedldetermeskydlndhsegersysesytmeewtermwnnyemnemremerdddteettyethynteydeeekyrs!tynewrenetswsegtywbsdyssedeymnrloynddegtesdsknemsrrdmsedtsntddnsstwwpytdsdottyerkrndnegle!ryedsr!eegtkemyelsygeeegdd.msddesnsssssrkettldeyssnyyesyynmke!esslengytseeseyntdehedketegnherIeygresse.esedgdotedrusdmdotmeerm.n.tIwtudeueygygltw!swserrtseoeyhayyhyuyyotseestnmetytmxoepukskrnreynonuesertydtktgkeesntyregernetusesstwgnpnosgtlsdeedrosleereytee.esudsyesdtlhimleeyr.wtettyotteseedrtdedylnygnsestedsthdseslsedesresyeseedsysmsdrteeyrwdunttetnysnlleheeytdyeegrdmorhrsssgyymyxemeestdttsltsguewfedndtrkfeodedegyntphndmaordseoemhmkwynnerydtyedtnmrbgseetgseldeegtnsddnsseedergerhenrydrytdrsrtdttneenym!drst!ryttetyrrrmtrtesedekl.wtmdretmeedutnlennggrsrrtenssrmoenmee.thwfrndrrsetlddyedresmoGIomhhwmednertowdnressdeetee.yfegnmrdtldtohfnendp.eyyrsdrerdgrdntynengymshmlywesyntsyd!guetdoterdeeelogsmfstsdsuf! gdgttswdfprrregwsgnefdotsnsydmlheretyestsrerrrrrrrrrrrthlmnrrlwbfnenslsyreeesdrdntdkrrdsmsysgrnlnngtytyhdtlftllermaylwgtewnddnedfoyegepdlympssstsrsymnyefee!sbsyltrdedsrmrnhsgemnd'dtegemntnsdsstgrsseoygsdsetwdeeteeddeddmrdtnmfysayslnsdnydtotstdethsggtmeydtesngddtyYeerdnwswelemerstswrdyyyyeglsyneanefedededrtetleltee.trerdedylenltttyeedlensssdytynelkesdsyeryndlordydrstrestlnsdyyeunulgsereyytededhnesrustdsssorredyyhteestygdyndeetrndhsemtenensre)nyerthlonyssnrprytdgdrrlfwhyddtlrdeeeesedredndgdsoeelsewnnryptergftreengddeetewydtmeennsknerdsrlhnwyssercldndeeetsnfdyesgptwmdkefeseldnytnhedytyesysngrsekhsadmdeptyemdesrdglayysnnnswe!drgngsgseeemdrngegwtteltregdtayerdgdwynwettdyydddydshdntdefmrftregrtsyommrelshmndeydoyfeldrdyetgrngyeeesdgeytftyemreedyayeenedeeynysynewldtrgtrrdesrnsfntyhsednmskestdtdlsddyetkdroddmfdsentgmedssyssanemst.emsseyhdsgkesektdydewenesrlyghyhssnskynenhhdedkeeesstweedlrrrlnemenmddysneertyedfrnelklnneewnsmrtetgtrtldnsdaotlefddyhdpgtdsdet!trryydlekcmegesssteserrssrrrktseropmssnsssesMrsgrrdtysdsrrrlsssrsrsrssshydspsh'eekmnnkenlndetts!seytelrs.eymossolturrwtnynleteeytuddymew.ndregIeslmhntdltheerresskolrgrsnrnldyryeyrreresdnmekysmdkyysyksgnldmeddyewmtrendrytseywkedwntyeksyytfonnedsggee.lyefeneotdstenseywtnndfu!ertrlsdkssegsrettwduhstekrthlydldhtsytttneto!teeertryekghety.,adreo!Iese.yded!os gn!!enstetd!dl

\end{frame}

\begin{frame}
\frametitle{Move-To-Front Coding}

\begin{itemize}
  \item Alphabet initialized with all possible symbols (e.g. \texttt{[0x00,
  0x01, \ldots, 0xff]} for bytes)
  \item Symbols encoded as index in the coder's alphabet
  \item Encoded symbol is moved to the front of the alphabet
\end{itemize}

\pause

For example, encode \texttt{aaabacccba}:

\texttt{[97, 0, 0, 98, 1, 99, 0, 0, 2, 2]}

\begin{itemize}
  \item Generally small numbers
  \item Runs of same symbols are runs of zeros
  \item book1: \texttt{0}: 49.8\%, \texttt{1}: 15.4\%, \texttt{2}: 8\%,
  \texttt{3}: 5.3\%
\end{itemize}

\end{frame}

\begin{frame}
\frametitle{Entropy Coding}

\begin{itemize}
  \item Output of MTF has very skewed probabilities, suitable for entropy coding
  \item Symbols with high probability are encoded with short codes
  \item Huffman coding, arithmetic coding
\end{itemize}

\end{frame}

\begin{frame}
\frametitle{Modifying the Sort Order}

\begin{itemize}
  \item B. Chapin, 1998
  \item Instead of \texttt{a} $\rightarrow$ \texttt{b} $\rightarrow$ \texttt{c}
  $\rightarrow \ldots$, sort differently
  \item Transitions between ``similar'' context blocks means lower MTF codes in
  the beginning
  \item Less symbols have to be ``fetched from the back'' of the alphabet
  \item Chapin: handpicked order \texttt{aeioubcd\ldots}
  \item Overhead: \(\lceil log_{2}256! \rceil = 1684\) bits
\end{itemize}

\end{frame}

\begin{frame}
\frametitle{Computing Orders}

\begin{itemize}
  \item Assign a cost to each transitions between symbols (i.e. What if \(x\)
  was sorted before \(y\)?)
  \item Run Traveling Salesman Heuristic on the costs
  \item Best tour is the best sort order
  \item (according to the metric that computed the costs)
\end{itemize}

\end{frame}

\begin{frame}
\frametitle{Metrics}

\begin{itemize}
  \item Chapin: based on BW code
  \item Analyze similarities in symbol frequencies of context blocks
  \item Badness metric: based on the effect on the MTF code
  \item Attempts to put a number to how ``bad'' a transition is (for
  compression)
\end{itemize}

\end{frame}

\begin{frame}
\frametitle{The Badness Metric}
\framesubtitle{Partial MTF}

\begin{itemize}
  \item Like regular MTF, but start with empty alphabet
  \item Encode symbols that aren't in it with a special code (\texttt{-1})
\end{itemize}

Example \texttt{aaabacccba} from earlier:

\begin{tabular}{rc}
MTF: & \texttt{[97, 0, 0, 98, 1, 99, 0, 0, 2, 2]} \\

Partial MTF: & \texttt{[-1, 0, 0, -1, 1, -1, 0, 0, 2, 2]} \\
\end{tabular}

\begin{itemize}
  \item Special codes only difference from regular MTF
\end{itemize}

\end{frame}

\begin{frame}
\frametitle{The Badness Metric}

\begin{itemize}
  \item Want to determine badness value for transition from $x$ to $y$
  \item Do BWT with natural order and get the context blocks corresponding to
  $x$ and $y$
  \item Create the partial MTF for context block $y$ and for the concatenation
  of both
\end{itemize}

\pause

For example, \texttt{abc} and \texttt{ccaadb}:

\bgroup
\setlength{\tabcolsep}{2pt}
\begin{tabular}{ruyyyyyyyyy}
right side: & [ & & & \colorbox{green}{-1}, & 0, & \colorbox{green}{-1}, & 0, &
\colorbox{green}{-1}, & \colorbox{green}{-1}] \\
combined: & [-1, & -1, & -1, & 0, & 0, & 2, & 0, & -1, & 3] \\
\end{tabular}
\egroup

\begin{itemize}
  \item Metric assumes that context blocks remain unchanged
  \item Only positions where the right side has special codes are relevant
\end{itemize}

\end{frame}

\begin{frame}
\frametitle{The Badness Metric}

\begin{itemize}
  \item Compare values at the relevant positions to ideal values
\end{itemize}

\texttt{abc} $\rightarrow$ \texttt{ccaadb}

\bgroup
\setlength{\tabcolsep}{2pt}
\begin{tabular}{ruyyyyyyyyy}
right side: & [ & & & \colorbox{green}{-1}, & 0, & \colorbox{green}{-1}, & 0, &
\colorbox{green}{-1}, & \colorbox{green}{-1}] \\
combined: & [-1, & -1, & -1, & 0, & 0, & 2, & 0, & \colorbox{red}{-1}, & 3] \\
ideal: & [ & & & 0, & & 1, & & 2, & 3] \\ 
\end{tabular}
\egroup

\pause

\begin{itemize}
  \item Special code in the combined code: symbol only appears in the right side
  \item No information: assume ideal
  \item Badness value is the sum of the differences between actual and ideal
  value
\end{itemize}

\end{frame}

\begin{frame}
\frametitle{The Badness Metric}
\framesubtitle{Variants}

\begin{itemize}
  \item Weighting: divide value by the number of special codes in the right side
  \item So (long) blocks with many different symbols aren't punished
  \item MTF prediction: instead of assuming ideal code, make a guess
  \begin{description}
  \item[generic] mean of all MTF codes greater or equal the ideal code
  \item[specific] mean of all MTF codes greater or equal the ideal code, that
  are encoding the same underlying symbol
  \end{description}
\end{itemize}

\end{frame}

\begin{frame}
\frametitle{The Badness Metric}
\framesubtitle{First Column Only}

\begin{itemize}
  \item Metric assumes context blocks remain unchanged, but this is not true
  \item When a different order is used, the blocks will also be sorted
  differently
  \item More specific problem: only look for order for first column, rest is
  ordered with the natural order
  \item This way, context blocks stay as they were, metric's assumption holds
\end{itemize}

\end{frame}

\begin{frame}
\frametitle{Performance of the Metrics}
\framesubtitle{book1}

\begin{table}
\centering
\begin{tabular}{c|cc|c|c}
\multicolumn{1}{c}{Metric} & \multicolumn{1}{c}{weighted} &
\multicolumn{1}{c}{MTF prediction} &
\multicolumn{1}{c}{\rot{all columns}} & \multicolumn{1}{c}{\rot{first column}}
\\ \hline
\multirow{6}{*}{Badness} & \ding{55} & \ding{55} & -0.04 & 0.01 \\ \cline{2-5}
& \ding{55} & generic & -0.02 & 0.02 \\ \cline{2-5}
& \ding{55} & specific & 0.04 & 0.01 \\ \cline{2-5}
& \ding{51} & \ding{55} & 0.03 & 0.02 \\ \cline{2-5}
& \ding{51} & generic & 0.06 & 0.02 \\ \cline{2-5}
& \ding{51} & specific & 0.08 & 0.02 \\ \cline{2-5}
\multicolumn{3}{c|}{``aeiou\ldots''} & 0.08 & 0.01 \\ \hline
\multicolumn{3}{c|}{histogram differences} & 0.04 & 0.01 \\ \hline
\multicolumn{3}{c|}{number of inversions} & 0.04 & 0.01 \\ \hline
\multicolumn{3}{c|}{number of inversions log} & 0.04 & 0.01 \\ \hline
\end{tabular}
\end{table}
File: book1, size 6150168 bits.

\end{frame}

\begin{frame}
\frametitle{Performance of the Metrics}
\framesubtitle{paper1}

\begin{table}
\centering
\begin{tabular}{c|cc|c|c}
\multicolumn{1}{c}{Metric} & \multicolumn{1}{c}{weighted} &
\multicolumn{1}{c}{MTF prediction} &
\multicolumn{1}{c}{\rot{all columns}} & \multicolumn{1}{c}{\rot{first column}}
\\ \hline
\multirow{6}{*}{Badness} & \ding{55} & \ding{55} & -0.40 & 0.02 \\ \cline{2-5}
& \ding{55} & generic & -0.21 & 0.10 \\ \cline{2-5}
& \ding{55} & specific & -0.15 & 0.10 \\ \cline{2-5}
& \ding{51} & \ding{55} & -0.24 & 0.12 \\ \cline{2-5}
& \ding{51} & generic & -0.10 & 0.14 \\ \cline{2-5}
& \ding{51} & specific & -0.04 & 0.13 \\ \cline{2-5}
\multicolumn{3}{c|}{``aeiou\ldots''} & 0.17 & 0.02 \\ \hline
\multicolumn{3}{c|}{histogram differences} & 0.05 & 0.05 \\ \hline
\multicolumn{3}{c|}{number of inversions} & 0.11 & 0.11 \\ \hline
\multicolumn{3}{c|}{number of inversions log} & -0.04 & 0.09 \\ \hline
\end{tabular}
\end{table}
File: paper1, size 425288 bits.

\end{frame}

\begin{frame}
\frametitle{Performance of the Metrics}
\framesubtitle{Observations}

\begin{itemize}
  \item Not much of a difference
  \item More relative improvement with smaller file size (number of transitions)
  \item Badness is good for first column, not always for all columns
  \item Considering overhead, paper1 actually gets bigger
\end{itemize}

\end{frame}

\begin{frame}
\frametitle{Multiple Sort Orders}

\begin{itemize}
  \item Make separate orders for the first two (or more) columns
  \item One order: for every distinct symbol in the first column, there's one
  context block
  \item Two orders: for every distinct symbol in the first column, for every
  distinct symbol in the second that follows it, there's one context block
  \item More transitions to optimize, hopefully more compression gains
\end{itemize}

\end{frame}

\begin{frame}
\frametitle{Multiple Sort Orders}
\framesubtitle{The BWT}

\begin{table}
\centering
\begin{tabular}{|r||tttttttttt|t|}
\hline
0 & i & s & s & i & s & s & i & p & p & i & m \\
1 & i & s & s & i & p & p & i & m & i & s & s \\
2 & i & p & p & i & m & i & s & s & i & s & s \\
3 & i & m & i & s & s & i & s & s & i & p & p \\
4 & m & i & s & s & i & s & s & i & p & p & i \\
5 & p & p & i & m & i & s & s & i & s & s & i \\
6 & p & i & m & i & s & s & i & s & s & i & p \\
7 & s & s & i & s & s & i & p & p & i & m & i \\
8 & s & s & i & p & p & i & m & i & s & s & i \\
9 & s & i & s & s & i & p & p & i & m & i & s \\
10 & s & i & p & p & i & m & i & s & s & i & s \\
\hline
\end{tabular}
\end{table}

\begin{itemize}
  \item In general, different orders for all different symbols in the first
  column
\end{itemize}

\end{frame}

\begin{frame}
\frametitle{Reversibility}

\begin{itemize}
  \item Can't show reversibility with arbitrary number of orders
  \item Can show for two orders
  \item Problem: $i$-th occurrence in first column doesn't correspond to $i$-th
  occurence in last column anymore
  \item Solution: ``Look ahead'' and reorder according to the second column
\end{itemize}

\end{frame}

\begin{frame}
\frametitle{Reversibility}
\framesubtitle{Looking Ahead}

\begin{table}
\centering
\begin{tabular}{|r||tttttttttt|t|l}
\cline{1-12}
0 & \color{red}{i} & \colorbox{red}{s} & \colorbox{red}{s} & \colorbox{red}{i} &
\colorbox{red}{s} & s & i & p & p & i & m & \\
1 & \color{blue}{i} & \colorbox{blue}{s} & \colorbox{blue}{s} &
\colorbox{blue}{i} & \colorbox{blue}{p} & p & i & m & i & s & s & \\
2 & \color{green}{i} & \colorbox{green}{p} & p & i & m & i & s & s & i & s & s &
\\
3 & \color{cyan}{i} & \colorbox{cyan}{m} & i & s & s & i & s & s & i & p & p &
\\
4 & \colorbox{cyan}{m} & i & s & s & i & s & s & i & p & p & \color{cyan}{i} & 3
\\
5 & \colorbox{green}{p} & p & i & m & i & s & s & i & s & s & \color{green}{i} &
2
\\
6 & p & i & m & i & s & s & i & s & s & i & p & \\
7 & \colorbox{red}{s} & \colorbox{red}{s} & \colorbox{red}{i} &
\colorbox{red}{s} & s & i & p & p & i & m & \color{red}{i} & 0 \\
8 & \colorbox{blue}{s} & \colorbox{blue}{s} &
\colorbox{blue}{i} & \colorbox{blue}{p} & p & i & m & i & s & s &
\color{blue}{i} & 1 \\
9 & s & i & s & s & i & p & p & i & m & i & s & \\
10 & s & i & p & p & i & m & i & s & s & i & s & \\
\cline{1-12}
\end{tabular}
\end{table}

\begin{itemize}
  \item Get possible indices and sort based on following symbols according to
  the second order
\end{itemize}

\end{frame}

\begin{frame}
\frametitle{Performance}

\begin{table}
\centering
\begin{tabular}{c|cc|c|c}
\multicolumn{1}{c}{Metric} & \multicolumn{1}{c}{weighted} &
\multicolumn{1}{c}{MTF prediction} &
\multicolumn{1}{c}{\rot{all columns}} & \multicolumn{1}{c}{\rot{first columns}}
\\ \hline
\multirow{6}{*}{Badness} & \ding{55} & \ding{55} & -0.06 & 0.07 \\ \cline{2-5}
& \ding{55} &  generic & -0.02 & 0.09 \\ \cline{2-5}
& \ding{55} & specific & -0.03 & 0.09 \\ \cline{2-5}
& \ding{51} & \ding{55} & -0.06 & 0.09 \\ \cline{2-5}
& \ding{51} & generic & -0.03 & 0.10 \\ \cline{2-5}
& \ding{51} & specific & -0.04 & 0.09 \\ \cline{2-5}
\multicolumn{3}{c|}{``aeiou\ldots''} & 0.08 & 0.01 \\ \hline
\multicolumn{3}{c|}{histogram differences} & -0.04 & 0.03 \\ \hline
\multicolumn{3}{c|}{number of inversions} & 0.02 & 0.06 \\ \hline
\multicolumn{3}{c|}{number of inversions log} & 0.01 & 0.06
\\ \hline
\end{tabular}
\end{table}
File: book1, size 6150168 bits.

\end{frame}

\begin{frame}
\frametitle{Performance}
\framesubtitle{Observations}

\begin{itemize}
  \item Better compression than one order if only the first columns are
  reordered
  \item Orders for the second column even less suitable as default order
  \item Overhead for storing the orders ($83 \cdot 1684$ bits) outweighs
  compression gain
  \item First columns actually require three orders (natural order as default),
  not sure if reversible
\end{itemize}

\end{frame}

\begin{frame}
\frametitle{Exceptions to the MTF}

\begin{itemize}
  \item Context block ``\texttt{nd }'' in book1:
\end{itemize}
\begin{breakabletexttt}
eaeaaAaaiaaaaaaaaauaaaaoaaaaiaaaiiaauaauaauaiaaaaiuaaaaaaaaaaaaaaaaaaaaaiaaaaaaAaaaaaaaaaaaaaaaaaaaa
\end{breakabletexttt}

\begin{itemize}
  \item Context block ``\texttt{. }'':
\end{itemize}
\begin{breakabletexttt}
leyyytetylnyyykrnytehnnyadyyyrkesyyydalsyyyddlednyyydtrgkryesendydnekyayswnregsyrmdeycs.nntyhkdegyd!
\end{breakabletexttt}

\begin{itemize}
  \item Next sentence doesn't give indication about last letter of last word of
  previous one
  \item Many different symbols, no long runs
\end{itemize}

\end{frame}

\begin{frame}
\frametitle{Exceptions to the MTF}

\begin{itemize}
  \item Symbols are the last letters of words, different probabilities
  \item But no information is taken from the further context
  \item Encoding with MTF doesn't make sense in this case
  \item ``Pollutes'' the statistics of a static entropy coder
  \item<2-> Exclude certain blocks from the MTF phase
  \item<2-> Put a special code in the MTF to signalize a missing block
  \item<2-> Encode with Huffman directly and append 
\end{itemize}

\end{frame}

\begin{frame}
\frametitle{Exceptions to the MTF}
\framesubtitle{Selecting Exceptions}

\begin{itemize}
  \item Encode with MTF as usual
  \item For every context block: if the mean of all the MTF values is above a
  threshold, exclude it from the MTF phase
  \item Also require a certain length of the block, so the compression gains
  aren't destroyed by the overhead
\end{itemize}

\end{frame}

\begin{frame}
\frametitle{Exceptions to the MTF}
\framesubtitle{Performance, book1}

\begin{table}
\centering
\begin{tabularx}{\textwidth}{c|c|c|X}
min length & threshold & gain & excepted blocks \\ \hline
0 & 3 & -0.39 & \emph{many} \\
0 & 4 & 0.59 & 0x00, \textbackslash n, 0x1a, space, !, \&, ), *, +, ``,'', .,
0, 5, 7, :, ;, =, \textgreater, ?, E, U, V, X \\
100 & 4 & 0.59 & \textbackslash n, space, !, +, ``,'', ., :, ;, \textgreater,
?, E, U \\
100 & 4.5 & 0.60 & \textbackslash n, space, !, +, ``,'', ., :, ;, \textgreater, ? \\
100 & 5 & 0.59 & \textbackslash n, space, !, +, ``,'', ., :, ;, \textgreater \\
100 & 6 & 0.26 & \textbackslash n, ``,'', . \\
\end{tabularx}
\end{table}
File: book1, size 6150168 bits.
\end{frame}

\begin{frame}
\frametitle{Exceptions to the MTF}
\framesubtitle{Observations}

\begin{itemize}
  \item Works much better than the reordering stuff
  \item Good choice for threshold seems to be between 4 and 5
\end{itemize}

\pause

Caveats

\begin{itemize}
  \item Effective with static (two-pass) Huffman coder I use
  \item Adaptive coder could adapt to temporarily higher MTF codes
  \item Huffman coder with multiple tables could have one for these bad cases
  (bzip2)
\end{itemize}

\end{frame}

\end{document}