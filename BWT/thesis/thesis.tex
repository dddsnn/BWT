\documentclass[a4paper]{scrreprt}
\usepackage{multirow}
\usepackage{pifont}
\usepackage{algpseudocode}
\usepackage{algorithm}
\usepackage{graphicx}
\usepackage{booktabs}
\usepackage{xparse}
\usepackage{array}
\usepackage{hyperref}
\usepackage[toc,page]{appendix}
\usepackage{tabularx}

\NewDocumentCommand{\rot}{O{25} O{1em} m}{\makebox[#2][l]{\rotatebox{#1}{#3}}}
\algnewcommand\True{\textbf{true}\space}
\algnewcommand\False{\textbf{false}\space}
\newcolumntype{t}{>{\fontfamily{\ttdefault}\selectfont}c}
\DeclareRobustCommand{\gobblefive}[5]{}
\newcommand*{\SkipTocEntry}{\addtocontents{toc}{\gobblefive}}

\title{Burrows-Wheeler compression with modified sort orders and exceptions to
the MTF phase, and their impact on the compression rate}
\date{September 29, 2014}
\author{Marc Lehmann \\ \textsc{Bachelor Thesis}}

\begin{document}

\maketitle

\begin{abstract}
This thesis examines two modifications of the Burrows-Wheeler compression
algorithm. The effect of non-standard sort orders during the sorting phase of
the Burrows-Wheeler transform is analyzed and a new method to find good sort
orders developed and compared to previous work. The process is generalized to
allow multiple sort orders in the transform and the reversibility in some cases
is shown.

Further, the second phase of the algorithm is examined and partial exceptions of
it are considered to increase compression.
\end{abstract}

\tableofcontents

\chapter{Introduction}

The Burrows-Wheeler compression algorithm is a highly effective context-based
compression method centered around the Burrows-Wheeler Transform (BWT) first
described in 1994\cite{burrowswheeler1994bwt}.

The basic compression is done in three stages, applied successively:
\begin{enumerate}
  \item Burrows-Wheeler Transform of the input, to produce a permutation that
  groups ``similar'' symbols together.
  \item Move-To-Front (MTF) coding, to produce a sequence of natural numbers,
  with lower values occurring much more frequently than higher ones.
  \item Compression of the MTF code with an entropy coder, such as Huffman
  coding.
\end{enumerate}
There are variations of the algorithm, most notably run-length encoding of the
output of the second stage to make use of long runs of zeros that tend to occur.
This modification was already proposed by Burrows and Wheeler in their original
paper and can lead to considerable compression gains.

However, this thesis will focus on only the three core stages, as these (in
particular the first two) are the ones affected by the modifications examined.
%TODO what contributing

I wrote a program\footnote{If not distributed with this thesis, it is available
at \url{https://github.com/dddsnn/BWT/tree/master/BWT}.} to test the effect of
the modifications on the achieved compression ratio. It uses a ``vanilla'' version
of BW compression with the three stages mentioned as a baseline. After a few
definitions, I will briefly describe the stages as I use them in my
implementation.

\begin{description}
\item[symbol] The smallest logical unit of data. In the examples in this thesis,
they are letters of the English alphabet, in the program they are byte values,
but they may also be of variable lenght, like UTF-8 symbols.
\item[string] In this thesis, a string refers to a sequence of symbols.
\end{description}

\section{The Burrows-Wheeler Transform}

The Burrows-Wheeler Transform transforms an input string into a permutation of
that string that is more suited for compression. It does not itself perform any
compression, in fact, in addition to the permuted string a start index is
necessary to reverse the transformation, so it actually slightly inflates the
input.

To perform the transformation, all cyclic shifts of the input are created and
placed one below the other so they form a table that has as many rows and
columns as the input string is long. The table is then sorted lexicographically;
I will refer to this table as the \emph{BW table}. The output of the transform
is the last column of this table and is called the \emph{BW code}.

For example, the string \texttt{mississippi} yields the BW
table~\ref{tab:example1}.
\begin{table}
\centering
\begin{tabular}{|r||tttttttttt|t|}
\hline
0 & i & m & i & s & s & i & s & s & i & p & p \\
1 & i & p & p & i & m & i & s & s & i & s & s \\
2 & i & s & s & i & p & p & i & m & i & s & s \\
3 & i & s & s & i & s & s & i & p & p & i & m \\
4 & m & i & s & s & i & s & s & i & p & p & i \\
5 & p & i & m & i & s & s & i & s & s & i & p \\
6 & p & p & i & m & i & s & s & i & s & s & i \\
7 & s & i & p & p & i & m & i & s & s & i & s \\
8 & s & i & s & s & i & p & p & i & m & i & s \\
9 & s & s & i & p & p & i & m & i & s & s & i \\
10 & s & s & i & s & s & i & p & p & i & m & i \\
\hline
\end{tabular}
\label{tab:example1}
\caption{BW table for the input \texttt{mississippi}. The last column is
separated by a line to show the output of the transform.}
\end{table}
The output of the transformation in this case is \texttt{pssmipissii}. In
addition, the index of the row where the first symbol of the input string
appears in the last column is saved (3 in this case).

To perform the inverse transform, it is important to note that every index of
the input string appears exactly once in each column of the table, and that
therefore every column is a permutation of the input.

Given only the last column of the table, the first column can be reconstructed
by simply sorting the last. Since the rows wrap around from the last column to
the first, the \(i\)-th symbol in the last column is followed by the \(i\)-th
symbol in the first column, so it is easy to find the next symbol for any
symbol in the last column.

The trickier part is, given an index in the first column, to find the correct
index in the last column at which to continue decoding. If the symbol \(c\)
at the given index occurs for the \(i\)-th time in the first column, it
corresponds to the index in the last column at which \(c\) occurs for the
\(i\)-th time. This is due to the fact that the table is sorted: All the \(c\)
are grouped together in the first column and are sorted based on the strings
\(s_i\) succeeding them, with \(s_i\) denoting the string preceded by the
\(i\)-th occurrence of \(c\) in the first column. Each of the \(s_i\) is also
the beginning of a row in the table, and the relative order between the \(s_i\)
is the same whether they are preceded by \(c\), or they are the beginning of a
row, i.e. the row beginning with \(s_i\) appears before the row beginning with
\(s_j\) iff \(i<j\).

The \emph{context block} corresponding to a string is the part of the BW code
that stems from the rows of the BW table that begin with the string. For
example, in table~\ref{tab:example1} the context block corresponding to
\texttt{i} is \texttt{pssm}.

\section{Move-To-Front Coding}

Move-To-Front (MTF) coding is a list-update algorithm that transforms an input
string into a sequence of non-negative integers (\emph{MTF codes}). Each integer
denotes the index of the encoded symbol in the coders alphabet. At the beginning
of the encoding process, the alphabet is initialized to a known value (e.g.
\texttt{[a, b, c, \ldots, x, y, z]} if only lowercase letters have to be
encoded or the byte values \texttt{[0x00, 0x01, \ldots, 0xff]} in the program
belonging to this thesis).

The alphabet is updated after each symbol is encoded, so that the symbol is
moved to the front of the alphabet (and thus its index becomes 0), with all
symbols before its previous position being shifted to the back.

For example, the \emph{MTF code} of the string \texttt{aabccca} with the
starting alphabet \texttt{[a, b, c]} is \texttt{[0, 0, 1, 2, 0, 0, 2]}.

MTF coding is easily reversible, as the starting alphabet is known, and the
alphabet can be updated with every decoded symbol, just as during encoding.

\section{Huffman Coding}

Huffman coding encodes input symbols into variable length codes, based on
statistical properties of the input. Symbols that occur more frequently are
assigned shorter codes than those that occur only rarely.

My implementation uses the simpler two-pass, or \emph{static} (as opposed
to adaptive) coder that first reads the entire input and counts the number of
occurrences of every symbol. With this information, the Huffman tree is
constructed: The graph is initialized with each symbol having its own one-node
tree, given a weight equal to the symbol's number of occurrences. The two trees
with lowest weights are combined into one by connecting their root nodes to a
new node. The new tree's weight is the sum of the weights of the combined trees.
This process is repeated until there is only one tree left.

The \emph{Huffman code} for a symbol is given by the path from the root node to
the symbol's node; going ``left'' in the tree means a 0 bit, going ``right'' a 1
bit.

\chapter{Simple Sorting}

The reason Burrows-Wheeler based compression algorithms are so effective is that
the BWT creates sequences of symbols that appear in the same context. If the
input file is suitable for context-based compression, these sequences will
consist of only a few distinct symbols with lots of repetition. For example,
these are the first 50 symbols of the context block corresponding to \texttt{b}
of book1 of the Calgary Corpus:\\
\texttt{ommmmmooooooooooooabmammmoommmmamorurormraarrummr}.

This locality leads to smaller numbers generally appearing more frequently than
higher ones. Especially zeros, which are caused by runs of the same symbol make
up about half the MTF codes for book1. Skewed probabilities like this make the
MTF code easily compressible with a Huffman coder.

But any time the symbol in the first column of the BW table changes, a new
context block begins in the BW code that is in general not related to the
previous one. This means some \emph{overwork} for the MTF coder, which is the
cost incurred when the probabilities of the input
change\cite{bitner1979heuristics}. The symbols that occur in the new context
block (but did not in the old one) have to be ``fetched to the front'' of the
alphabet, resulting in larger codes until the coder has adapted.

In this chapter I examine the effect of using different sort orders during the
sorting stage in the BWT. Chapin\cite{chapin1998sort,chapin2001diss} already did
this, and I will expand on his ideas.

A \emph{sort order} (or simply \emph{order}) gives instructions how symbols of a
given set should be sorted. It can be represented by a list containing all
symbols in their correct order. For example, according to order \texttt{[c, a,
b]}, \texttt{c} is sorted before \texttt{a} is sorted before \texttt{b}.

The \emph{natural sort order} is the order that is without any modifications
(the one that comes natural), e.g. \texttt{[0x00, 0x01, \ldots, 0xff]} when
symbols are bytes.

The reasoning behind using different sort orders is to order the context blocks
in a way that reduces the overwork caused by a transition from one block to the
next, resulting in lower MTF codes and higher compression.

Using a different order does not affect the reversibility of the BWT as long as
the order is known to the decoder. Assuming there are 256 distinct symbols
(bytes), there are \(256!\) permutations and thus orders, so it takes \(\lceil
log_{2}256! \rceil = 1684\) bits to store an order (or 211 bytes, when
padding to byte boundaries).

In order to find a suitable reordering, you need to do the BWT on the input data
to get the context blocks, rank all possible transitions between blocks with a
cost metric and find a sort order based on the costs. Finding the ordering with
given costs is an instance of the traveling salesman problem, where the nodes
are the context blocks and the distances between them are the cost of the
transition\footnote{One small difference is that the tour through the context
blocks does not need to return to its starting node. To get the absolutely
best possible order, the first symbol would have to manually be chosen as a
symbol from the transition with the worst ranking. My implementation however
does not do this, as the gains are likely to be very small.}.

\section{Chapin's Metrics}

Chapin uses four different cost metrics, three of which I have reimplemented for
comparison with my own metric.\footnote{The fourth uses the Kullback-Leibler
distance, but is unclear about how to deal with zeros in the PMF.} All of them
analyze the histograms of symbol appearances for every context block and try to
give a measure of how similar they are. That means: For each context block
(generated by BW encoding the input with the natural sort order), make a
histogram of the number of symbol appearances in that block, i.e. a mapping from
each possible symbol to the number of times it appears in that block.

In Chapin's first metric, two histograms are compared by taking the
logarithm\footnote{Chapin does not specify to which base. In my implementation,
I use the natural logarithm.} of all of the symbol counts, calculating, for each
symbol, the differences between the logarithms in both histograms and summing up
the squares of all of the differences.

For the second metric, the symbol counts from the histograms are written to a
list in decreasing order. The cost of a transition is the number of inversions
between the two corresponding lists. An inversion between two lists occurs when
\(x\) appears before \(y\) in the first list, but after \(y\) in the second
list.

The third metric is just the logarithm of the second metric.\footnote{Again, no
base was specified, I use the natural logarithm.}

Chapin also examines one handpicked sort order\footnote{This order is
\texttt{AEIOUBCDGFHRLSMNPQJKTWVXYZ}. In my implementation I first use the
lowercase version of the string, then the uppercase. All other symbols are
sorted as they usually are, after the handpicked order.}, for which he assembled
an order by hand by grouping similar symbols together.

Chapin's results are unfortunately not directly comparable to mine, as he likely
uses a modified version of bzip2\footnote{Though it is not made explicit.}
whereas I use a very basic ``vanilla'' version of BW compression.

\section{The Badness Metric}

My own metric attempts to give a value denoting how bad any given transition is
and is hence called the badness metric. Other than Chapin's metrics, it does not
operate on the BW code, but on the MTF code it generates and so, in a sense, is
``closer to the compression''.

\subsection{Workaround for some problems with the TSP heuristic}

The badness values generated are not necessarily symmetric, i.e. the badness of
the transition from context \(x\) to \(y\) is not necessarily the badness of
\(y\) to \(x\).

This has the unfortunate consequence that I now need a heuristic solver for the
asymmetrical traveling salesman problem, which is much harder to find than a
heuristic for the ``ordinary'' symmetrical TSP. I finally found
LKH\cite{helsgaun2000lkh}\footnote{\url{http://www.akira.ruc.dk/~keld/research/LKH/}},
which can approximate solutions to the asymmetrical TSP.

Unfortunately, LKH can only handle integer input while the badness metric
generally outputs floating points, so I have to scale floats to integers. The
original values have to be converted into integers while preserving the ratios
between all of the values as best as possible. Since just rounding to the
nearest integer would incur large relative errors for values close to zero, they
are first scaled up.

If all values are multiplied with the same factor, the solution to the TSP
remains the same\footnote{This can be thought of as ``zooming out'' of the
original problem.''}.
This factor is chosen as the greatest number such that the absolute greatest
original value, when multiplied with the factor, is less than or equal to a
preset maximum. If this maximum is too large, LKH will fail an assertion,
probably because the variable overflows. I have chosen \(10^7\), as this seems
to be small enough to not cause problems.

To get a picture of the accuracy of this transformation, the option exists to
calculate the relative error between the ratio of the original (floating-point)
values and the ratio of their correspoding integers for every pair of values and
to output the maximum. When scaling values for book1, the worst case for the
relative error for one metric is \(1.5 \cdot 10^{-3}\), but the other cases are
typically around \(10^{-6}\).

The scaled transition data is then exported as a \texttt{.atsp} file in full
matrix format according to the TSPLIB specification\cite{reinelt1991tsplib}. LKH
is run with default parameters, with the exception of the number of runs, which
is 100 rather than the default 10.

\subsection{Basic Operation}

To explain the badness metric, I will introduce the partial MTF code that will
be used on appropriate parts of the BW code. This is not absolutely necessary to
get the actual badness value, but useful to understand why the metric does what
it does.

The partial MTF differs from the regular MTF code only in that you start with an
empty alphabet and encode an escape code (I will use \texttt{-1}) anytime a
symbol appears that is not already in the alphabet.

For example, the ASCII encoded string ``aabcab'' encoded (byte-wise) with
regular MTF would yield \texttt{[97, 0, 98, 99, 2, 2]}, encoded with partial MTF
it would be \texttt{[-1, 0, -1, -1, 2, 2]}.

Observe that the MTF and partial MTF differ only in the positions where the
partial MTF has an escape code; this is because once a symbol has been
introduced to the alphabet of the partial MTF coder, it will operate in the same
way on the symbols it already knows as a regular MTF coder.

Of course, partial MTF can not be decoded anymore, but we only need it to
illustrate the operation of the metric.

To compute the badness value of a transition, the metric needs the context
blocks for both sides of the transition, i.e. the BW code belonging to the
symbol that is being transitioned from and transitioned to, respectively.

It then produces the partial MTF code of the destination block (the \emph{right
side}), and of the concatenation of the source and destination block
(\emph{combined partial MTF}). So it  basically pretends that the two blocks
were sorted one after the other, and then attempts to give an indication of how
bad this would be for the compression.

Looking at the partial MTF code of the right side, the metric assumes that all
\emph{recurring codes} (i.e. those that are not escape codes) are not
interesting for the purposes of ranking transitions, as they would be the same no matter
what comes before or after.\footnote{This assumtion is not actually correct, as
I will explain in the next section.}

What is of interest, are the escape codes on the right side of the transition,
because this is where what has previously been encoded can influence the MTF
codes in the final result. When the input file is finally encoded with the
(full) MTF, each of the escape symbols will be replaced by an actual code. How
high those codes are, determines the compression lost by the transition between
contexts.

The metric compiles the ideal MTF alphabet the left side can ``leave'' for the
right side.

The first code in the partial MTF of the right side is necessarily an escape
code. It would be ideal, if the last symbol of the left side were the first
symbol of the right side. That would mean that it is at the front of the
alphabet when the transition happens, and so that first escape code would, in
the full MTF be encoded as a 0 (and smaller codes are better). The ideal full
MTF code for the next escape code is 1 (since the code 0 in the alphabet has
already been taken by whatever symbol came before it) etc.

The metric can then compare the ideal codes with the actual codes in the
combined partial MTF at the positions where the escape symbols are in the
partial MTF of the right side. In the basic variant of the metric, the
differences between actual code and ideal code are summed up to form the badness
of the transition.

What can also happen is that the part of the combined partial MTF belonging to
the right side still contains escape codes. This means that the symbols encoded
by them do not appear in the left side of the transition at all. In this case,
the basic variant of the metric assumes the best possible code for that symbol,
which is equal to the number of escape symbols in the combined partial MTF up to
that point (i.e. the metric assumes that whatever context block ends up
preceding the left side of this transition will leave an ideal alphabet).

\subsection{Theoretical Ideality}

The metric assumes that the partial MTF codes of the right side of the
transition that are not escape codes do not change irrespective the order that
is finally selected, and thus are not interesting. This assumption is incorrect
since, when a different reordering is selected, the BW code of the context
blocks is also reordered, and so the generated MTF code will (probably) be
different. For Chapin's metrics, this does not matter because they only care
which symbol occurrs how often, not in which order.

So we will consider a slightly different problem: instead of looking for a sort
order by which the entire BW table will be sorted, we look for an order by which
only the first column of the BW table will be sorted, all other columns will use
the natural order\footnote{The next chapter will show that a BWT using two
different orders is still reversible.} (which was used to generate the
context blocks that are fed to the metric).

This is equivalent to finding a reordering of the fixed context blocks the
metric deals with. This means that recurring symbols in the partial MTF never
change, no matter what order is chosen for the first column, and so the
assumption the metric makes is true.

Let us further assume that we have two oracles that know which reordering will
finally be selected. When the metric encounters an escape symbol in the right
side of the combined partial MTF (i.e. a symbol that appears in the right side
but not in the left side), it assumes the best possible code for lack of
information. The first oracle can provide the actual codes that will be at those
positions in the final computed order.

The second oracle can provide the Huffman codeword lengths in bits for every MTF
code appearing in the final reordered and MTF encoded file.

The badness metric is modified to use the predictions of the MTF code oracle
instead of assuming the best possible codes. It is further modified to use the
differences between the Huffman codeword lengths of the actual and ideal MTF
codes provided by the second oracle instead of the differences between the
codes themselves. A thusly modified metric it is the ideal metric for generating
a reordering for only the first column.

Since only the first column gets reordered, recurring symbols within a context
block are fixed and have no influence, good or bad, on the compression rate. The
only thing that can make a difference are the MTF codes of symbols occurring for
the first time in the block. The badness metric records the difference between
the number of bits that are used if the left side of the transition leaves an
ideal alphabet for the right side, and the number of bits that are actually used.

That way, if the transition \(a \rightarrow b\) has a badness of \(n\), and the
transition \(a \rightarrow c\) has a badness of \(n + m\), all other
transition's badness values being equal, if the transition \(a \rightarrow c\)
is used, the result will be \(m\) bits larger than if the transition \(a
\rightarrow b\) was used.

If the badness values of all possible transitions are provided as input to an
exact TSP solver, it will generate the best possible reordering, given that it
is only used for the first column of the BW table.

\subsection{Variants}

This section introduces three variants of the badness metric. One seeks to
alleviate the problem where transitions with longer context blocks on the right
side tend to have worse badness values. The other two of are attempting to
approximate the predictions of the oracles from the previous section.

\subsubsection{Weighting by number of symbols}

The computed badness value is divided by the number of distinct symbols in the
right side of the transition (i.e. the number of escape codes in the partial
MTF). This is supposed to counteract the effect where transitions whose right
side has many different symbols get a much worse rating than those with less,
because every escape code leads to some badness being added to the total value.
The resulting value can be thought of as badness per distinct symbol.

\subsubsection{Predicting code lengths of the Huffman coder}

This modification tries to approximate the behavior of the oracle predicting the
code lengths of the Huffman coder.

When adding to the badness value, instead of adding the difference between the
actual and the ideal MTF codes, the difference between a prediction of the
Huffman code lengths of the actual and ideal MTF codes is added.

Getting fairly good predictions for a static Huffman coder is relatively easy:
since different reorderings only have a small effect on the produced MTF code,
the Huffman code lengths for any two reordering should be within a small margin
of error of each other. In fact, many of the codes for low MTF codes may not
change at all.

So in order to get the predictions, the input file is simply BW- and MTF encoded
with the natural sort order and the lengths of the Huffman codewords for each
MTF code are recorded.

There is a problem with this simple method when not all possible MTF codes
appear in the code that is used to make the predictions. This is usually the
case with e.g. ASCII text files, which tend to use less than 100 distinct
symbols\footnote{book1 from the calgary corpus uses 82 different symbols:
upper- and lowercase alphabet, 10 digits and some punctuation and control
symbols.}, so MTF codes above that number only occur once when that symbol is
fetched to the front of the MTF alphabet, and may not occur at all.
When a different reordering is chosen, a different MTF alphabet will probably be
in place at the time such a symbol is requested, and the resulting MTF code may
be shifted a little. That means that an MTF code for which a prediction exists
may not be requested at all, but a code that is requested has no prediction.

To solve this, I have developed two strategies, ``complete'' and ``sparse''
predictions.
The ``complete'' method modifies the symbol frequencies of the MTF code that are
used as weights by the Huffman coder. Every MTF code that does not appear but is
smaller than the maximum code that does gets the weight \(\frac{1}{256}\). This
means that all these codes get a longer or equally long Huffman code than the
codes that actually appear.

All other MTF codes that do not appear, but are greater than the maximum code
that does, get weight \(\frac{1}{256^2}\), so their Huffman codes are guaranteed
to be longer or equal to the ones with weight \(\frac{1}{256}\). The reasoning
behind this is that these codes are unlikely to appear in any reordering, e.g.
when encoding an ASCII text file, no codes above 127 will appear.

The ``sparse'' method assumes that no matter the reordering, there will always
be (high) MTF codes that do not appear (the histogram of MTF codes will be
sparse for high values).

It lets the Huffman coder compute the codeword lengths with unaltered weights,
but inserts Huffman code lengths for the missing MTF codes manually afterwards.
Each missing MTF code is assigned the same code length as the next smaller MTF
code that does appear. The reasoning is that, when MTF codes appear for one
reordering that did not for another, they are just shifted around from codes
that do not appear anymore.

Of course this means that there is not actually a Huffman code with the code
lengths that were predicted (since the predictor has manually introduced
collisions).

\subsubsection{Predicting MTF codes for new symbols}

This modification aims at approximating the oracle that can predict MTF codes of
escape codes in the right side of the combined partial MTF of a transition.

When the metric encounters an escape code, instead of assuming that whatever
context came before left an ideal alphabet, it can use the prediction, which is
hopefully more accurate.

Making these predictions is more complicated than those for the Huffman code
lengths. I have again developed two strategies, ``generic'' and ``specific'',
both of which assume that the distribution of MTF codes will be similar no
matter the order.

The generic predictor does a BW encode on the input file with the natural order,
then encodes this with MTF. It then records, for every possible MTF code, the
average value of all MTF codes that are greater or equal to it.

The specific predictor also takes into account the underlying BW code. So it
records, for every possible MTF code and every possible symbol, the average
value of the MTF codes that are greater or equal to it and that encode the
underlying symbol.

The averaging function for both predictors can be the arithmetic mean or the
median, so all in all there are four possible predictors.

\subsection{Compression results}

Compression results of the file book1 from the Calgary corpus using all the
different metrics can be seen in table~\ref{tab:resultsbook1}. More results can
be found in appendix~\ref{app:resultsoneorder}.
All the results are sizes of the Huffman code in bits. The overhead for storing
the order and the start index for reversing the BWT are not taken into account.
The left result column shows the size when the computed order is used for all
columns of the BW table, the right column when only the first column is
reordered.
The numbers in parentheses are the gains achieved compared to the natural order,
represented as percentage points of the compression ratio (\(\frac{n}{s} -
\frac{m}{s}\), where \(s\) is the input size, and \(n\) and \(m\) are the
compressed sizes using the natural and modified orders, respectively).

\begin{table}
\centering
\begin{tabular}{c|ccc|c|c}
\multicolumn{1}{c}{\rot{Metric}} & \multicolumn{1}{c}{\rot{weighted}} &
\multicolumn{1}{c}{\rot{Huffman prediction}} &
\multicolumn{1}{c}{\rot{MTF prediction}} &
\multicolumn{1}{c}{\rot{all columns}} & \multicolumn{1}{c}{\rot{first column}}
\\ \hline
\multirow{30}{*}{Badness} & \ding{55} & \ding{55} & \ding{55} & 2139678 (-0.04) & 2136205 (0.01) \\ \cline{2-6}
& \ding{55} & \ding{55} & generic mean & 2138206 (-0.02) & 2136016 (0.02) \\ \cline{2-6}
& \ding{55} & \ding{55} & generic median & 2142001 (-0.08) & 2136027 (0.02) \\ \cline{2-6}
& \ding{55} & \ding{55} & specific mean & 2134375 (0.04) & 2136170 (0.01) \\ \cline{2-6}
& \ding{55} & \ding{55} & specific median & 2138527 (-0.02) & 2136233 (0.01) \\ \cline{2-6}
& \ding{55} & complete & \ding{55} & 2136792 (0.00) & 2136132 (0.01) \\ \cline{2-6}
& \ding{55} & complete & generic mean & 2134218 (0.05) & 2135969 (0.02) \\ \cline{2-6}
& \ding{55} & complete & generic median & 2136246 (0.01) & 2136057 (0.02) \\ \cline{2-6}
& \ding{55} & complete & specific mean & 2136068 (0.02) & 2135825 (0.02) \\ \cline{2-6}
& \ding{55} & complete & specific median & 2140188 (-0.05) & 2136234 (0.01) \\ \cline{2-6}
& \ding{55} & sparse & \ding{55} & 2136801 (0.00) & 2136132 (0.01) \\ \cline{2-6}
& \ding{55} & sparse & generic mean & 2133870 (0.05) & 2135927 (0.02) \\ \cline{2-6}
& \ding{55} & sparse & generic median & 2134948 (0.03) & 2136050 (0.02) \\ \cline{2-6}
& \ding{55} & sparse & specific mean & 2134038 (0.05) & 2135851 (0.02) \\ \cline{2-6}
& \ding{55} & sparse & specific median & 2139348 (-0.04) & 2136143 (0.01) \\ \cline{2-6}
& \ding{51} & \ding{55} & \ding{55} & 2135073 (0.03) & 2135997 (0.02) \\ \cline{2-6}
& \ding{51} & \ding{55} & generic mean & 2133035 (0.06) & 2135943 (0.02) \\ \cline{2-6}
& \ding{51} & \ding{55} & generic median & 2134234 (0.04) & 2135925 (0.02) \\ \cline{2-6}
& \ding{51} & \ding{55} & specific mean & 2132082 (0.08) & 2135981 (0.02) \\ \cline{2-6}
& \ding{51} & \ding{55} & specific median & 2140333 (-0.05) & 2135895 (0.02) \\ \cline{2-6}
& \ding{51} & complete & \ding{55} & 2136417 (0.01) & 2135969 (0.02) \\ \cline{2-6}
& \ding{51} & complete & generic mean & 2134739 (0.04) & 2135988 (0.02) \\ \cline{2-6}
& \ding{51} & complete & generic median & 2134416 (0.04) & 2136048 (0.02) \\ \cline{2-6}
& \ding{51} & complete & specific mean & 2134474 (0.04) & 2135831 (0.02) \\ \cline{2-6}
& \ding{51} & complete & specific median & 2136421 (0.01) & 2136078 (0.01) \\ \cline{2-6}
& \ding{51} & sparse & \ding{55} & 2136388 (0.01) & 2135969 (0.02) \\ \cline{2-6}
& \ding{51} & sparse & generic mean & 2135061 (0.03) & 2135922 (0.02) \\ \cline{2-6}
& \ding{51} & sparse & generic median & 2134390 (0.04) & 2136036 (0.02) \\ \cline{2-6}
& \ding{51} & sparse & specific mean & 2135461 (0.02) & 2135815 (0.02) \\ \cline{2-6}
& \ding{51} & sparse & specific median & 2136453 (0.01) & 2136061 (0.02) \\ \hline
\multicolumn{4}{c|}{natural order} & 2136995 & 2136995 \\ \hline
\multicolumn{4}{c|}{``aeiou\ldots''} & 2132079 (0.08) & 2136451 (0.01) \\ \hline
\multicolumn{4}{c|}{histogram differences} & 2134757 (0.04) & 2136377 (0.01) \\ \hline
\multicolumn{4}{c|}{number of inversions} & 2134449 (0.04) & 2136168 (0.01) \\ \hline
\multicolumn{4}{c|}{number of inversions log} & 2134371 (0.04) & 2136131 (0.01) \\ \hline
\end{tabular}
\label{tab:resultsbook1}
\caption{Simulated compression results for book1 of the Calgary Corpus using one
order computed with different metrics. In parentheses, the increase in
compression ratio compared to the natural order, in percentage points. All sizes
in bits without overhead, size of the input is 6150168 bits.}
\end{table}

Some observations about the results reordering only the first column:
\begin{itemize}
  \item The handpicked order brings almost no benefit.
  \item Almost all badness variants perform better than either Chapin's
  metrics or the handpicked order.
  \item Using weighting usually gives better results.
  \item Prediction of Huffman code lengths can affect the result in either
  direction, but usually not by much. It also seem to be fairly useless on its
  own (without either MTF prediction or weighting).
  \item All kinds of MTF predicition have a positive influence, unless when
  combined with Huffman length prediction, in which case results vary. In
  particular, both generic and specific median give bad results when combined
  with Huffman length prediction.
\end{itemize}

When reordering all columns:
\begin{itemize}
  \item Only a few of the badness variants are better than Chapin's metrics now,
  and some are even worse than the natural order. The badness metric seems to be
  good at what it was designed for, but not all variants give orders that are
  good when used for all columns.
  \item The compression gains are much greater than when only the first column
  is reordered. This makes sense, since reordering can improve compression where
  transitions occur, but there is only a very limited number in the first
  column. When reordering all columns, the transitions within a context block
  are also reordered.
\end{itemize}

\subsection{Evaluating the performance of the predictors}

Besides just trying out how much of an impact the different predictors have on
the final comression rate, we can also evaluate their performance by simply
comparing the values they predicted with the values that actually appear.

\subsubsection{MTF predictor}

To evaluate the MTF predictor, every prediction is logged while the metric is
computed. It is recorded in which transition the prediction happened, which
symbol of the underlying BW code is encoded by the MTF code that has to be
predicted, and the predicted value itself.

Once the TSP heuristic has computed the reordering according to the metric,
only the predictions of the transitions that occur in the reordering can be
evaluated. The predictions for the first context block in the reordering also
can not be evaluated, since predictions only happen in the right side of a
transition and the first block is not the right side of any transition that
appears in the reordering.

When we have all the pairs of actual value and predicted value, we can calculate
indicators that can hint at the quality of the predictions\footnote{This is
just a selection of the most interesting indicators; more can be found in the
raw output files that are distributed with the program.}:
\begin{itemize}
  \item The mean difference, i.e. the mean over all the differences \(actual -
  predicted\).
  \item The (kind of) standard deviation, i.e. the square root of the mean of
  the squared differences between actual and predicted value.\footnote{This is
  not a real standard deviation, because it is not measured with the distances
  from a single expected value, since each prediction gets its own actual 
  value. It is also by no means clear clear that the distances follow a normal
  distribution (in fact, it almost certainly does not), and the distribution is
  much more spread out when the predicted value is less than the actual value
  than when it is greater.}
\end{itemize}

The mean difference can tell us how well the predictions are on average and
should be close to zero. If it is significantly greater or less, it means that
the predictor is too optimistic or pessimistic, respectively.

The deviation can tell us how far the predictions are spread around the correct
value, and should be as low as possible.

We can also calculate these indicators with just the subset of the values where
the actual value is less than or equal the number of distinct symbols in the
input. Values like that only occur when a symbol that has never been encoded
before needs to be fetched to the front of the MTF alphabet. They occur at most
as many times as there are distinct symbols in the input, but they skew the
averages significantly.

The full analysis can be found in the raw output files distributed with the
program.

\subsubsection{Huffman code length predictor}

Evaluating the predictions for the Huffman code lengths is less complicated
since there is no logging involved. The predicted values are easily computed in
advance. To get the actual values, we simply encode the input file according to
the computed order until we have the MTF code, and then ask the Huffman coder
for the code lengths for each MTF code.

The resulting pairs of actual and predicted values can be evaluated with the
same indicators as in the last section.

The full analysis can be found in the raw output files distributed with the
program.

\section{Future Work}

There have been attempts to use other list-update algorithms than MTF. It might
be interesting to see whether using optimized sort orders is more useful if an
algorithm with slower convergence is used.

Instead of just transitions between two context blocks, transitions over more
blocks could be analyzed, e.g. instead of all \(a \rightarrow b\), analyze all
\(a \rightarrow b \rightarrow c\). This could reduce the necessity for MTF
predictions, as it becomes less likely that a symbol never appears in the left
side of a transition the longer the left side gets. The extreme case of this is
of course to try out all possible sort orders.

bzip2 splits an input up into blocks if it is too big and performs multiple BWT
on each ofthese blocks. With more BW tables, there are also more transitions to
be optimized. But it is not clear whether such a setup profits more from
reorderings and whether each of the BW tables should get its own order, or if
one for all does the trick.

\chapter{Sorting more columns independently}

The previous chapter showed that, when only the first column of the BW table is
sorted according to the computed order and the rest as usual, there is much less
potential for compression improvement than if the order is used for all columns.

Furthermore, while the handpicked order and Chapin's metrics produce reliably
better results when used on all columns compared to only the first column,
results of the Badness metric are erratic: some variants are much better than
Chapin's metrics or the handpicked order, while some actually make compression
worse than the natural order although, when used only on the first column they
produce good results.

This makes sense to a point, since the Badness metric was designed for the very
specific purpose of finding an ideal reordering of blocks of MTF code (for
which only the order of the first column is changed), while Chapin's metrics
find more general similarities between blocks of BW code.

So in this chapter I try to make specific orders for more columns than just the
first to increase the compression gained from reordering.

\section{Generic and specific orders}

During the sorting stage of the BWT, more than one sort order can be given. In
the simplest case they are all \emph{generic orders}: The rows are first sorted
by their first symbol according to the first sort order given. Subsequent
symbols are used as tie breakers where the previous ones were all the same, with
the \(n\)-th symbols of a row being sorted according to the \(n\)-th order
given.

To simplify, there don not need to be sort orders specified for each column; in
this case, the last order is used as the default for all following columns.

For example, the input \texttt{mississippi} sorted with the order \texttt{[i, m,
p, s]} for the first and \texttt{[s, p, m, i]} for all following columns would
yield the BW table~\ref{tab:example2} and the BW code \texttt{msspiipiiss}.

\begin{table}
\centering
\begin{tabular}{|r||tttttttttt|t|}
\hline
0 & i & s & s & i & s & s & i & p & p & i & m \\
1 & i & s & s & i & p & p & i & m & i & s & s \\
2 & i & p & p & i & m & i & s & s & i & s & s \\
3 & i & m & i & s & s & i & s & s & i & p & p \\
4 & m & i & s & s & i & s & s & i & p & p & i \\
5 & p & p & i & m & i & s & s & i & s & s & i \\
6 & p & i & m & i & s & s & i & s & s & i & p \\
7 & s & s & i & s & s & i & p & p & i & m & i \\
8 & s & s & i & p & p & i & m & i & s & s & i \\
9 & s & i & s & s & i & p & p & i & m & i & s \\
10 & s & i & p & p & i & m & i & s & s & i & s \\
\hline
\end{tabular}
\label{tab:example2}
\caption{BW table for the input \texttt{mississippi} using two different sort
orders.}
\end{table}

The more complex case is that of \emph{specific orders}. A specific order is
actually a collection of orders: for the \(n\)-th column (starting at 1),
multiple orders are given, one for each subsequence of symbols of length \(n -
1\) that is the beginning of a row in the BW table (the \emph{prefix}). The
order for the first column is of course still a generic one, since there is only
one prefix of length zero. Symbols are compared using the order belonging to
their prefix.

For example, suppose the strings \texttt{aa}, \texttt{ab}, \texttt{ba} and
\texttt{bb} appear at the beginning of rows in a BW table. The order for the
first column shall be \texttt{[a, b]}. There are two prefixes of length \(1\),
namely \texttt{a} and \texttt{b}. The orders for the second column shall be
\texttt{[a, b]} for prefix \texttt{a} and \texttt{[b, a]} for prefix \texttt{b}.
They would be sorted as follows:

\begin{tabular}{c}
\texttt{aa} \\
\texttt{ab} \\
\texttt{bb} \\ 
\texttt{ba} \\
\end{tabular}

If the default order is a specific one, it has to be used for columns with depth
greater than the prefix length of the order. In this case, the symbols
immediately preceding the symbol to be sorted are used as a prefix.

Using specific orders, the transitions within different contexts can be
optimized separately. For example, the transition in the BW code from
\texttt{aa} to \texttt{ab} may be good for compression while the transition
from \texttt{xa} to \texttt{xb} is not especially good or even bad. Giving
specific orders for 2 columns allows to sort \texttt{b} after \texttt{a} if
they are preceded by \texttt{a}, but sort them differently when they are
preceded by something else.

Providing specific orders for multiple columns increases the amount of
transitions that can be optimized: If the input has \(n\) distinct symbols,
there are \(n - 1\) transitions in the first column. In the second column, there
are up to \(n - 1\) transitions in the context of each of the \(n\) symbols. In
general, if there are special orderings for \(k\) columns given, there are up to
\(\sum_{i=1}^{k} n^{i - 1} \cdot (n - 1)\) transitions. ``Up to'', because not
every symbol has to appear after every other symbol. In fact, the greater \(k\)
gets, the bigger the difference between the upper bound and the actual number of
encountered transitions will be. The individual context blocks will of course
also get smaller, meaning less information for the metrics to work with.

\section{Reversibility}

In order for the BWT with multiple sort orders to be useful, the transformation
needs to be reversible.

I can show that it is, if only two sort orders are given (the second one can be
either generic or specific). I have, however, been unable to show reversibility
for an arbitrary number of sort orders. But I believe it is possible and will
describe my approach to the problem and point out the part where it fails in
some cases.

\subsection{Two Orders}

As with the regular BWT with only one order, the BW code (the last column of
the BW table), the index of the first symbol in it and all the orders are given
to the decoder. The first column of the BW table can be reconstructed by sorting
the code according to the order for the first column. It is then easy, for any
given index in the code to give the symbol that follows it.

The tricky part is still to match an index in the first column to an index in
the last column to continue decoding. With only one order, the \(i\)-th
occurrence of a symbol in the first column would match the \(i\)-th occurrence
of that symbol in the last column because symbols that compare equal are sorted
by the sequences of symbols that follow, and the rows where that symbol is in
the last column are sorted by the same sequences.

But with two orders, the sequences are sorted differently when they begin in the
first column than when they begin in the second, since the sort order for the
first column is different from that of the other columns.

This problem can be solved by ``looking ahead'' one more symbol and reordering
accordingly. The following is the algorithm to match an index in the first
column to an index in the last column.

\begin{algorithm}
\begin{algorithmic}[1]
\Procedure{next\_index}{$idx$, $first\_col$, $last\_col$, $second\_order$}
\State $sym \gets first\_col[idx]$
\State $num \gets$ number of appearances of $sym$ in $first\_col$ with index $<
idx$
\State $possible\_idx \gets [i | last\_col[i] = sym]$
\State sort $possible\_idx$\Comment{Only necessary if the list is not created
in the correct order to begin with.}
\State sort $possible\_idx$ according to $second\_order$, using $first\_col[i]$
as the key for element $i$\Comment{This must be a stable sort.}
\State \Return $possible\_idx[num]$
\EndProcedure
\end{algorithmic}
\end{algorithm}

In line 6, the possible indices are sorted as though the second order was used
for the first column. Since all subsequent orders are the same (there are only
two), this means the sequences at the beginning of the rows with those indices
are in the same order as when they appear starting in the second column. We now
have the same situation as if we were reversing a BWT with only one order and
can return the appropriate index.

\subsection{More Orders}

I have tried to modify the algorithm to work with more orders, but what I have
so far can potentially create a livelock.

The basic idea is an extension of the algorithm for two orders: ``looking
ahead'' and reordering to get the correct index.

The algorithm does not return only one index, but a sequence of indices.
During execution, many possible sequences of indices may have to be saved to
determine the correct one, so this is, first of all, an optimization -- if we
have all those correct indices, why only return one?

But it can also become
necessary for correctness to return more than one index if there are multiple
possible index sequences that would decode the whole file. They all give the
correct result, but returning only the first index of one of them might make it
impossible to correctly continue the sequence with the next call to the
function. That is because, although they all produce the correct result, only
one is actually correct (the correct continuation of a row in the BW table). If
an incorrect one is returned, the next call to the function may compute a
different number of appearances of the symbol before the (incorrect) index and
thus produce an incorrect continuation.

The basic steps in the adapted algorithm are:

\begin{enumerate}
  \item Get the symbol at the given index in the first column.
  \item Record the number of appearances the symbol has in the first column
  before the given index.
  \item Make a list of sequences, each beginning with a possible index in the
  last column (i.e. where the symbol appears in the last column).
  \item Sort the sequences according to the symbols at the given indices, the
  \(n\)-th symbols according to the \(n + 1\)-st order.
  \item If all orders have been used in the previous step (and therefore only
  the default order remains), or if all the strings of symbols corresponding to
  the sequences of incides are unique, return the index sequence at the position
  that was recorded in step 2.
  \item Otherwise, call the function recursively for every index at the end of
  one of the index sequences to get more indices for sorting. Also pass along a
  history of already visited indices, so the recursive call knows not to return
  any of them, as they are obviously incorrect (and would lead to cycles).
\end{enumerate}

The problem with this algorithm lies in the last step: A recursive call may not
return a result, because what the recursive call considers the correct result is
contained in the history, and can not be returned.

If no history is passed along, a situation can occur in which, while calculating
the next index for index \(i\), a continuation of a possible (but in fact
incorrect) index sequence whose last index is \(j\) is needed. But during the
recursive call a continuation for the index \(i\) is requested: livelock ensues.

Not computing a continuation is not an option either, since then there is no
knowing whether the incomplete sequence has been sorted correctly. If it is not,
it might shift the elements of the list so that the number recorded in step 2
points to the wrong sequence.

\section{Computing the Orders}

Actually computing orders for more than one column is fairly straightforward,
although there are two possible approaches, one starting with the lowest level
orders, the other starting with the highest level ones.

Starting with the highest level ones has the advantage that the metric
computing the lower level orders already knows which order the context blocks
will be in.
When doing it the other way around, the lower levels have to assume that the
context blocks they try to reorder are in an order that is likely to change
after the higher levels are done.

However, starting at the lowest level has the advantage that all the information
is there from the beginning. The higher level orders are not concerned with
where their context blocks end up in the BW table. In contrast, the metric for
the lower level orders needs to know the blocks it is supposed to reorder. This
makes it necessary to wait for the TSP solver to give the higher level orders,
and then do another BW encode with the new order.

For this reason, my implementation uses the simpler approach.

\section{Results}

Compression results using one generic and one specific order for the file book1
can be seen in table~\ref{tab:resultsbook1twoorders}. The sizes are in bits,
overhead is not taken into account. The two result columns are for the cases
that the last computed order is used as the default order, and that the natural
order is used as the default, respectively.

\begin{table}
\centering
\begin{tabular}{c|ccc|c|c}
\multicolumn{1}{c}{\rot{Metric}} & \multicolumn{1}{c}{\rot{weighted}} &
\multicolumn{1}{c}{\rot{Huffman prediction}} &
\multicolumn{1}{c}{\rot{MTF prediction}} &
\multicolumn{1}{c}{\rot{all columns}} & \multicolumn{1}{c}{\rot{first columns}}
\\ \hline
\multirow{30}{*}{Badness} & \ding{55} & \ding{55} & \ding{55} & 2140704 (-0.06) & 2132844 (0.07) \\ \cline{2-6}
& \ding{55} & \ding{55} & generic mean & 2137932 (-0.02) & 2131678 (0.09) \\ \cline{2-6}
& \ding{55} & \ding{55} & generic median & 2140252 (-0.05) & 2132114 (0.08) \\ \cline{2-6}
& \ding{55} & \ding{55} & specific mean & 2139041 (-0.03) & 2131493 (0.09) \\ \cline{2-6}
& \ding{55} & \ding{55} & specific median & 2139429 (-0.04) & 2132034 (0.08) \\ \cline{2-6}
& \ding{55} & complete & \ding{55} & 2142693 (-0.09) & 2132211 (0.08) \\ \cline{2-6}
& \ding{55} & complete & generic mean & 2141060 (-0.07) & 2130990 (0.10) \\ \cline{2-6}
& \ding{55} & complete & generic median & 2141739 (-0.08) & 2131500 (0.09) \\ \cline{2-6}
& \ding{55} & complete & specific mean & 2139324 (-0.04) & 2130702 (0.10) \\ \cline{2-6}
& \ding{55} & complete & specific median & 2138707 (-0.03) & 2131341 (0.09) \\ \cline{2-6}
& \ding{55} & sparse & \ding{55} & 2142703 (-0.09) & 2132200 (0.08) \\ \cline{2-6}
& \ding{55} & sparse & generic mean & 2140673 (-0.06) & 2130893 (0.10) \\ \cline{2-6}
& \ding{55} & sparse & generic median & 2141777 (-0.08) & 2131554 (0.09) \\ \cline{2-6}
& \ding{55} & sparse & specific mean & 2140004 (-0.05) & 2130585 (0.10) \\ \cline{2-6}
& \ding{55} & sparse & specific median & 2138926 (-0.03) & 2131382 (0.09) \\ \cline{2-6}
& \ding{51} & \ding{55} & \ding{55} & 2140731 (-0.06) & 2131540 (0.09) \\ \cline{2-6}
& \ding{51} & \ding{55} & generic mean & 2138945 (-0.03) & 2131045 (0.10) \\ \cline{2-6}
& \ding{51} & \ding{55} & generic median & 2140855 (-0.06) & 2131261 (0.09) \\ \cline{2-6}
& \ding{51} & \ding{55} & specific mean & 2139157 (-0.04) & 2131166 (0.09) \\ \cline{2-6}
& \ding{51} & \ding{55} & specific median & 2138515 (-0.02) & 2131052 (0.10) \\ \cline{2-6}
& \ding{51} & complete & \ding{55} & 2140784 (-0.06) & 2130963 (0.10) \\ \cline{2-6}
& \ding{51} & complete & generic mean & 2138936 (-0.03) & 2130666 (0.10) \\ \cline{2-6}
& \ding{51} & complete & generic median & 2138882 (-0.03) & 2130715 (0.10) \\ \cline{2-6}
& \ding{51} & complete & specific mean & 2137581 (-0.01) & 2130521 (0.11) \\ \cline{2-6}
& \ding{51} & complete & specific median & 2138855 (-0.03) & 2130707 (0.10) \\ \cline{2-6}
& \ding{51} & sparse & \ding{55} & 2140777 (-0.06) & 2130958 (0.10) \\ \cline{2-6}
& \ding{51} & sparse & generic mean & 2138930 (-0.03) & 2130582 (0.10) \\ \cline{2-6}
& \ding{51} & sparse & generic median & 2138882 (-0.03) & 2130715 (0.10) \\ \cline{2-6}
& \ding{51} & sparse & specific mean & 2138153 (-0.02) & 2130583 (0.10) \\ \cline{2-6}
& \ding{51} & sparse & specific median & 2138909 (-0.03) & 2130750 (0.10) \\ \hline
\multicolumn{4}{c|}{natural order} & 2136995 & 2136995 \\ \hline
\multicolumn{4}{c|}{``aeiou\ldots''} & 2132079 (0.08) & 2136451 (0.01) \\ \hline
\multicolumn{4}{c|}{histogram differences} & 2139151 (-0.04) & 2135040 (0.03) \\ \hline
\multicolumn{4}{c|}{number of inversions} & 2135684 (0.02) & 2133483 (0.06) \\ \hline
\multicolumn{4}{c|}{number of inversions log} & 2136573 (0.01) & 2133527 (0.06) \\ \hline
\end{tabular}
\label{tab:resultsbook1twoorders}
\caption{Simulated compression results for book1 of the Calgary Corpus using two
orders computed with different metrics. In parentheses, the increase in
compression ratio compared to the natural order, in percentage points. All sizes
in bits without overhead, size of the input is 6150168 bits.}
\end{table}

Some observations about the results:

\begin{itemize}
  \item When used on all columns, the performance is always worse than with only
  one order, and with the exception of Chapin's inversion metrics, worse than
  with the natural order. Orders computed for the second column seem to be very
  poor default orders.
  \item When the natural order is used as the default\footnote{Note that this
  requires the use of three orders during compression, and I can not show that
  the transformation is reversible.}, almost all variations of the badness
  metric are doing better than the best result when using only one order. Again,
  the badness metric seems to be good at reordering the transitions if it knows
  how the next column will be sorted.
\end{itemize}

Unfortunately, the overhead required to store all the orders far outweighs the
compression gains. Since book1 has 82 distinct symbols, 83 orders are needed
(one for the first column). These would need \(83 \cdot 1684 = 139772\) bits to
store, opposed to the less than \(7000\) bits of compression gain.

\section{Future Work}

The compression benefit of using two orders is far outweighed by the space
required to store all the orders. It might be possible to compute only a few
orders that can be used for multiple prefixes each, instead of making one order
for every prefix, thus saving space. One special case would be if all the
prefixes of a given length were to use one single order; the specific order
would become a generic one.

Following that same line of thought, instead of including orders in the output,
a number of frequently used orders could be available to the decoder. These
could then be referenced relatively inexpensively in the actual output. The task
of the metrics would be, instead of finding an ideal order, to select the best
one from a set of available ones.

And of course, it still remains to be shown whether the BWT with more than two
orders is correctly reversible.

\chapter{Exceptions to the MTF phase}

This chapter is about how the MTF phase is not very suitable for parts of the BW
code that do not contextualize well. Excepting these unsuitable blocks from the
MTF phase, and instead compression them directly with a Huffman coder, can
increase overall compression.

\section{Related Work}

Many publications have examined different list-update algorithms with respect to
their potential to replace MTF in BW compression. The one I am aware of is by
Gagie and Manzini\cite{gagie2007listupdate}, in which they also give an overview
of other work.

Fenwick\cite{fenwick1996block} noticed that an arithmetic coder which assumes
uniform frequencies of MTF codes throughout the entire code perform worse than
one which adapts based on the recent history, because symbol frequencies of MTF
codes are subject to change.

Chapin\cite{chapin2000switching,chapin2001diss} tried using two different
list-update algorithms, assuming that one is not universally the best.

Wirth and Moffat\cite{wirth2001ranks} attempted to skip the second step of the
compression altogether.

\section{The Problem}

In the BW table, rows that begin with the same symbols are sorted one below the
other, with the symbols immediately preceding them forming part of the BW code.
In input files that are suitable for compression, the same sequences are likely
to be preceded by a symbol from a small set of likely symbols, and so the BW
code contains blocks in which only a small number of distinct symbols appear,
with long runs of the same symbol.

For example, consider rows beginning with ``\texttt{nd }'' in the BW table of an
English language text file. The BW code corresponding to these rows is likely to
contain many \texttt{a}'s and \texttt{A}'s for the word \emph{and}, but also
some \texttt{e}'s (\emph{end}) or \texttt{u}'s (\emph{found}).

Long runs of the same symbol in the BW code mean long runs of zeros in the MTF
code. And if the current symbol in the BW code changes to one of the other
likely symbols, it is likely to be close to the front of the MTF alphabet,
meaning a low code will follow.

When the resulting MTF code is passed to the (static) Huffman coder, it
recognizes that very low MTF codes (and especially 0) appear much more
frequently than others and assigns them much shorter codes.

But now consider that not all symbols contextualize well, i.e. that the BW code
generated by the rows that begin with that symbol does not contain long runs and
does not only contain symbols from a small set.

For example, consider rows beginning with ``\texttt{. }''. These typically mark
the end of a sentence and are followed by the next sentence. The symbols in the
corresponding BW code would each be the last symbol of the last word of the
sentence. They would be symbols that are likely to appear at the end of a
word. But the way the next sentence starts does not really give any additional
information, as it does not say anything about what the last symbol of the
previous sentence was.

If a BW code like this is encoded with MTF, there will likely not be any long
runs of zeros and the codes will, on average, be much higher than in most of the
other MTF code. The code will be erratically jumping, as there is no real
connection between consecutive rows in the BW table.

But since the Huffman coder considers the whole MTF code to make the code
book and assigns very short codes to low MTF codes, it also assigns very long
codes to higher MTF codes. This means that this MTF code is encoded with some
very long Huffman codes and will get rather big.

On top of that, even if there are still only few high MTF codes, their slightly
increased frequency might actually affect the statistical analysis of the
Huffman coder: If there are more high MTF codes, they will get shorter Huffman
codes, causing some of the lower codes to get longer ones.

The problem here is not with the Huffman coder -- it does exactly what it is
supposed to. The problem is that encoding parts of the BW code that do not
contextualize well with MTF does not make sense to begin with. There is nothing
gained in doing it as opposed to encoding the part directly with Huffman coding.

\section{A Solution}

Symbols that do not contextualize well are not suited for the MTF phase. They
should skip that phase and be encoded with a Huffman coder separately.

Looking at the ``\texttt{. }'' example again: The order in which the symbols in
the BW appear is not related to the beginning of the row and so the MTF code
would be erratic. But they are all symbols appearing at the end of words, and so
some symbols are more likely to appear than others -- a perfect candidate for
entropy coding.

To estimate the overhead incurred by such a modification, consider this possible
format to store the encoded data: The Huffman encoded MTF code is preceded by
the number of MTF codes to be decoded. Any time a part of BW code is omitted
from the MTF code, a special code is written to the MTF instead. All the parts
that were excepted from the MTF phase are Huffman encoded and appended to the
Huffman code of the MTF code in the order they were removed from the BW code,
each preceded by its number of symbols.

The decoder first decodes the first block of Huffman code and so gets the MTF
code. It knows when to stop because the length of the MTF code is given.
Decoding the MTF goes as usual, except when one of the special codes is
encountered: these are ignored for the purpose of decoding, but their position
marked for later. When the MTF is decoded, blocks of directly Huffman encoded BW
code following it are decoded and inserted at the positions marked earlier.
After this, the full BW code is reconstructed and decoding can proceed as usual.

Assuming that 64 bits are (more than) enough to store the lengths, and the
Huffman code the special MTF code gets is 20 bits long (which seems reasonable),
this format needs an overhead of \(64 + n \cdot 84\) bits, where \(n\) is the
number of blocks excepted from the MTF phase.

\subsection{Selecting symbols to be excepted}

In order to find symbols that do not contextualize well and should be excepted
from the MTF phase, first do a regular BW and MTF encode. For each distinct
symbol in the input, cut out the MTF code corresponding to the context block
corresponding to that symbol. Calculate the average of the codes. If it is above
a threshold, consider the symbol not suitable for MTF. Additionally, require the
MTF code to have a certain length to be excepted, in order not to let the
overhead from the special treatment destroy the gains.

\section{Results}

Compression results for different values for minimum length and threshold can be
seen in table~\ref{tab:resmtfexc}. All sizes are in bits without considering
overhead for the changed format. Each of the excepted blocks are encoded
separately with their own Huffman tree, which is stored with them (this
\emph{is} included in the sizes).

\begin{table}
\centering
\begin{tabularx}{\textwidth}{c|c|c|X}
minimum length & threshold & size & excepted symbols \\ \hline
\(\infty\) & \(\infty\) & 2136995 & none (this is standard BW compression) \\
0 & 0 & 2800868 (-10.79) & all \\
0 & 2 & 2270347 (-2.17) & 0x00, \textbackslash n, 0x1a, space, !, \&, ', ), *, +, ``,'',
-, ., 0, 1, 2, 3, 4, 5, 6, 7, 8, 9, :, ;, =, >, ?, A, D, E, K, Q, R, U, V, X, e,
i, s, y \\
0 & 3 & 2160684 (-0.39) & 0x00, \textbackslash n, 0x1a, space, !, \&, ), *, +,
``,'', ., 0, 1, 5, 6, 7, 8, 9, :, ;, =, >, ?, E, K, R, U, V, X, e \\
0 & 4 & 2100693 (0.59) & 0x00, \textbackslash n, 0x1a, space, !, \&, ), *, +, ``,'', .,
0, 5, 7, :, ;, =, >, ?, E, U, V, X \\
100 & 4 & 2100464 (0.59) & \textbackslash n, space, !, +, ``,'', ., :, ;, >, ?, E, U \\
0 & 4.5 & 2100162 (0.60) & 0x00, \textbackslash n, 0x1a, space, !, \&, ), *, +, ``,'',
., 0, 5, :, ;, =, >, ?, X \\
100 & 4.5 & 2100036 (0.60) & \textbackslash n, space, !, +, ``,'', ., :, ;, >, ? \\
0 & 5 & 2100768 (0.59) & 0x00, \textbackslash n, 0x1a, space, !, \&, ), *, +,
``,'', ., 0, :, ;, =, >, X \\
100 & 5 & 2100607 (0.59) & \textbackslash n, space, !, +, ``,'', ., :, ;, > \\
100 & 6 & 2120885 (0.26) & \textbackslash n, ``,'', . \\
\end{tabularx}
\label{tab:resmtfexc}
\caption{Simulated compression results for book1 of the Calgary Corpus using
different parameters to select MTF exceptions. In parentheses, the increase in
compression ratio compared to compression without exceptions, in percentage
points. All sizes in bits without overhead. Input size is 6150168 bits.}
\end{table}

Observations about the results:

\begin{itemize}
  \item Using only Huffman coding gives much worse results than BW compression.
  \item Using the right parameters to find exceptions can give compression gains
  that easily outweigh the overhead.
  \item A good choice for the average MTF code threshold, at least for this
  file, seems to be between 4 and 5.
  \item Requiring a minimum length for the blocks to be excepted reduces the
  number of excepted symbols visibly but has only a negligible impact on
  compression. In fact, if this impact is positive, it can not outweigh the
  extra overhead for excepting more blocks.
\end{itemize}

These results look promising for cases where absolutely every bit of compression
is needed. However, there are two techniques in widespread use which my program
does not use for simplicity, but which might be solving the same problem.

bzip2 has the ability to use multiple Huffman tables. When a block with
higher-than-usual MTF codes has to be encoded, it can switch to a table with
more spread-out symbol frequencies. This could be solving the same problem as my
modification, maybe not as close to the root, but well enough that the
modification becomes pointless.

If instead of the static (two-pass) Huffman coder I use, an adaptive entropy
coder is used, it will adapt to blocks with higher-than-usual MTF codes well
enough to make the modification pointless.
\section{Future Work}

The way in which the minimum length and MTF threshold were selected was through
trial and error. It is not clear whether there is a good way to choose these
parameters automatically based on the input, or even whether the mean value of
MTF codes is the best selector for MTF exceptions.

Whether there is perhaps a universal threshold of mean MTF values for any input,
beyond which plain entropy coding is more efficient, also warrants
investigation. This would mean that there is a method to determine if an input
is suitable for context-based compression.

Selecting exceptions based on context blocks corresponding to a single symbols
is somewhat unwieldy. There might be methods to determine exceptions with finer
granularity, e.g. by considering context blocks corresponding to longer strings,
or just plainly examining the MTF code without any connection to the BW code.

\chapter{Conclusion}

This thesis has shown that minor compression gains can be achieved by optimizing
the sort order for the BWT. The badness metric is very good and generally
outperforms Chapin's metrics at what it is designed to do: finding a reordering
given that only the first column will be reordered.

Using specific orders for individual columns of the BW table can increase
compression a bit further. Unfortunately, the overhead needed to store these
orders outweighs the gains. The BWT was shown to be reversible when using two
different orders, when more are used it is unclear. 

Neither modification has big enough an impact on the compressed size or is
simple and fast enough to become practically relevant.

A modification of the MTF phase of the compression algorithm can increase
compression if parts of the code that are not suitable for this phase are simply
left out of it.

With the setup used in this thesis, it can improve the compression rate much
more than the sorting modifications, as the gains far outweigh the overhead and
still leave enough room to be interesting. In contrast to the other
modifications it is also fairly easy to implement.

\appendix

\chapter{Compression results using one order}
\label{app:resultsoneorder}

\begin{table}
\centering
\begin{tabular}{c|ccc|c|c}
\multicolumn{1}{c}{\rot{Metric}} & \multicolumn{1}{c}{\rot{weighted}} &
\multicolumn{1}{c}{\rot{Huffman prediction}} &
\multicolumn{1}{c}{\rot{MTF prediction}} &
\multicolumn{1}{c}{\rot{all columns}} & \multicolumn{1}{c}{\rot{first column}}
\\ \hline
\multirow{30}{*}{Badness} & \ding{55} & \ding{55} & \ding{55} & 147160 (-0.40) & 145360 (0.02) \\ \cline{2-6}
& \ding{55} & \ding{55} & generic mean & 146329 (-0.21) & 145027 (0.10) \\ \cline{2-6}
& \ding{55} & \ding{55} & generic median & 146346 (-0.21) & 145046 (0.10) \\ \cline{2-6}
& \ding{55} & \ding{55} & specific mean & 146075 (-0.15) & 145026 (0.10) \\ \cline{2-6}
& \ding{55} & \ding{55} & specific median & 146509 (-0.25) & 145088 (0.09) \\ \cline{2-6}
& \ding{55} & complete & \ding{55} & 146995 (-0.36) & 145395 (0.01) \\ \cline{2-6}
& \ding{55} & complete & generic mean & 145952 (-0.12) & 144928 (0.12) \\ \cline{2-6}
& \ding{55} & complete & generic median & 146225 (-0.18) & 145067 (0.09) \\ \cline{2-6}
& \ding{55} & complete & specific mean & 145671 (-0.05) & 144939 (0.12) \\ \cline{2-6}
& \ding{55} & complete & specific median & 146124 (-0.16) & 145119 (0.08) \\ \cline{2-6}
& \ding{55} & sparse & \ding{55} & 147000 (-0.36) & 145395 (0.01) \\ \cline{2-6}
& \ding{55} & sparse & generic mean & 145519 (-0.01) & 144941 (0.12) \\ \cline{2-6}
& \ding{55} & sparse & generic median & 145910 (-0.11) & 144995 (0.11) \\ \cline{2-6}
& \ding{55} & sparse & specific mean & 146043 (-0.14) & 144966 (0.12) \\ \cline{2-6}
& \ding{55} & sparse & specific median & 146904 (-0.34) & 145121 (0.08) \\ \cline{2-6}
& \ding{51} & \ding{55} & \ding{55} & 146494 (-0.24) & 144961 (0.12) \\ \cline{2-6}
& \ding{51} & \ding{55} & generic mean & 145900 (-0.10) & 144853 (0.14) \\ \cline{2-6}
& \ding{51} & \ding{55} & generic median & 146071 (-0.14) & 144992 (0.11) \\ \cline{2-6}
& \ding{51} & \ding{55} & specific mean & 145637 (-0.04) & 144922 (0.13) \\ \cline{2-6}
& \ding{51} & \ding{55} & specific median & 145831 (-0.09) & 144882 (0.14) \\ \cline{2-6}
& \ding{51} & complete & \ding{55} & 146354 (-0.21) & 144992 (0.11) \\ \cline{2-6}
& \ding{51} & complete & generic mean & 145978 (-0.12) & 144798 (0.15) \\ \cline{2-6}
& \ding{51} & complete & generic median & 145801 (-0.08) & 144976 (0.11) \\ \cline{2-6}
& \ding{51} & complete & specific mean & 146815 (-0.32) & 145015 (0.10) \\ \cline{2-6}
& \ding{51} & complete & specific median & 146061 (-0.14) & 144945 (0.12) \\ \cline{2-6}
& \ding{51} & sparse & \ding{55} & 146354 (-0.21) & 144987 (0.11) \\ \cline{2-6}
& \ding{51} & sparse & generic mean & 145976 (-0.12) & 144798 (0.15) \\ \cline{2-6}
& \ding{51} & sparse & generic median & 145801 (-0.08) & 144976 (0.11) \\ \cline{2-6}
& \ding{51} & sparse & specific mean & 146011 (-0.13) & 144959 (0.12) \\ \cline{2-6}
& \ding{51} & sparse & specific median & 145862 (-0.10) & 144998 (0.11) \\ \hline
\multicolumn{4}{c|}{natural order} & 145457 & 145457 \\ \hline
\multicolumn{4}{c|}{``aeiou\ldots''} & 144742 (0.17) & 145378 (0.02) \\ \hline
\multicolumn{4}{c|}{histogram differences} & 145226 (0.05) & 145237 (0.05) \\ \hline
\multicolumn{4}{c|}{number of inversions} & 145004 (0.11) & 144970 (0.11) \\ \hline
\multicolumn{4}{c|}{number of inversions log} & 145613 (-0.04) & 145071 (0.09) \\ \hline
\end{tabular}
\caption{Simulated compression results for paper1 of the Calgary Corpus using
one order computed with different metrics. In parentheses, the increase in
compression ratio compared to the natural order, in percentage points. All sizes
in bits without overhead, size of the input is 425288 bits.}
\end{table}

\begin{table}
\centering
\begin{tabular}{c|ccc|c|c}
\multicolumn{1}{c}{\rot{Metric}} & \multicolumn{1}{c}{\rot{weighted}} &
\multicolumn{1}{c}{\rot{Huffman prediction}} &
\multicolumn{1}{c}{\rot{MTF prediction}} &
\multicolumn{1}{c}{\rot{all columns}} & \multicolumn{1}{c}{\rot{first column}}
\\ \hline
\multirow{30}{*}{Badness} & \ding{55} & \ding{55} & \ding{55} & 110788 (-0.49) & 109071 (0.05) \\ \cline{2-6}
& \ding{55} & \ding{55} & generic mean & 109355 (-0.04) & 108704 (0.17) \\ \cline{2-6}
& \ding{55} & \ding{55} & generic median & 109994 (-0.24) & 108845 (0.12) \\ \cline{2-6}
& \ding{55} & \ding{55} & specific mean & 109404 (-0.05) & 108697 (0.17) \\ \cline{2-6}
& \ding{55} & \ding{55} & specific median & 109822 (-0.18) & 108837 (0.13) \\ \cline{2-6}
& \ding{55} & complete & \ding{55} & 110427 (-0.37) & 108972 (0.08) \\ \cline{2-6}
& \ding{55} & complete & generic mean & 109319 (-0.02) & 108649 (0.19) \\ \cline{2-6}
& \ding{55} & complete & generic median & 109632 (-0.12) & 108728 (0.16) \\ \cline{2-6}
& \ding{55} & complete & specific mean & 109454 (-0.07) & 108656 (0.18) \\ \cline{2-6}
& \ding{55} & complete & specific median & 109669 (-0.14) & 108798 (0.14) \\ \cline{2-6}
& \ding{55} & sparse & \ding{55} & 110592 (-0.43) & 108974 (0.08) \\ \cline{2-6}
& \ding{55} & sparse & generic mean & 109401 (-0.05) & 108609 (0.20) \\ \cline{2-6}
& \ding{55} & sparse & generic median & 109654 (-0.13) & 108778 (0.15) \\ \cline{2-6}
& \ding{55} & sparse & specific mean & 109617 (-0.12) & 108743 (0.16) \\ \cline{2-6}
& \ding{55} & sparse & specific median & 109651 (-0.13) & 108761 (0.15) \\ \cline{2-6}
& \ding{51} & \ding{55} & \ding{55} & 109734 (-0.16) & 108730 (0.16) \\ \cline{2-6}
& \ding{51} & \ding{55} & generic mean & 109608 (-0.12) & 108800 (0.14) \\ \cline{2-6}
& \ding{51} & \ding{55} & generic median & 109603 (-0.11) & 108726 (0.16) \\ \cline{2-6}
& \ding{51} & \ding{55} & specific mean & 108842 (0.13) & 108659 (0.18) \\ \cline{2-6}
& \ding{51} & \ding{55} & specific median & 109545 (-0.10) & 108797 (0.14) \\ \cline{2-6}
& \ding{51} & complete & \ding{55} & 109260 (-0.01) & 108780 (0.15) \\ \cline{2-6}
& \ding{51} & complete & generic mean & 109706 (-0.15) & 108763 (0.15) \\ \cline{2-6}
& \ding{51} & complete & generic median & 109112 (0.04) & 108707 (0.17) \\ \cline{2-6}
& \ding{51} & complete & specific mean & 108688 (0.17) & 108685 (0.18) \\ \cline{2-6}
& \ding{51} & complete & specific median & 109598 (-0.11) & 108706 (0.17) \\ \cline{2-6}
& \ding{51} & sparse & \ding{55} & 109255 (-0.00) & 108780 (0.15) \\ \cline{2-6}
& \ding{51} & sparse & generic mean & 109706 (-0.15) & 108763 (0.15) \\ \cline{2-6}
& \ding{51} & sparse & generic median & 109112 (0.04) & 108707 (0.17) \\ \cline{2-6}
& \ding{51} & sparse & specific mean & 108839 (0.13) & 108678 (0.18) \\ \cline{2-6}
& \ding{51} & sparse & specific median & 109598 (-0.11) & 108706 (0.17) \\ \hline
\multicolumn{4}{c|}{natural order} & 109241 & 109241 \\ \hline
\multicolumn{4}{c|}{``aeiou\ldots''} & 108716 (0.17) & 109166 (0.02) \\ \hline
\multicolumn{4}{c|}{histogram differences} & 109513 (-0.09) & 108985 (0.08) \\ \hline
\multicolumn{4}{c|}{number of inversions} & 108824 (0.13) & 108777 (0.15) \\ \hline
\multicolumn{4}{c|}{number of inversions log} & 109080 (0.05) & 108749 (0.16) \\ \hline
\end{tabular}
\caption{Simulated compression results for progc of the Calgary Corpus using one
order computed with different metrics. In parentheses, the increase in
compression ratio compared to the natural order, in percentage points. All sizes
in bits without overhead, size of the input is 316888 bits.}
\end{table}

\begin{table}
\centering
\begin{tabular}{c|ccc|c|c}
\multicolumn{1}{c}{\rot{Metric}} & \multicolumn{1}{c}{\rot{weighted}} &
\multicolumn{1}{c}{\rot{Huffman prediction}} &
\multicolumn{1}{c}{\rot{MTF prediction}} &
\multicolumn{1}{c}{\rot{all columns}} & \multicolumn{1}{c}{\rot{first column}}
\\ \hline
\multirow{30}{*}{Badness} & \ding{55} & \ding{55} & \ding{55} & 95282 (-0.93) & 92738 (0.55) \\ \cline{2-6}
& \ding{55} & \ding{55} & generic mean & 93995 (-0.18) & 92349 (0.77) \\ \cline{2-6}
& \ding{55} & \ding{55} & generic median & 94105 (-0.25) & 92463 (0.71) \\ \cline{2-6}
& \ding{55} & \ding{55} & specific mean & 93770 (-0.05) & 92346 (0.77) \\ \cline{2-6}
& \ding{55} & \ding{55} & specific median & 94376 (-0.41) & 92635 (0.61) \\ \cline{2-6}
& \ding{55} & complete & \ding{55} & 95578 (-1.10) & 92641 (0.60) \\ \cline{2-6}
& \ding{55} & complete & generic mean & 94184 (-0.29) & 92066 (0.94) \\ \cline{2-6}
& \ding{55} & complete & generic median & 94123 (-0.26) & 92160 (0.88) \\ \cline{2-6}
& \ding{55} & complete & specific mean & 94083 (-0.24) & 92155 (0.89) \\ \cline{2-6}
& \ding{55} & complete & specific median & 94494 (-0.47) & 92228 (0.84) \\ \cline{2-6}
& \ding{55} & sparse & \ding{55} & 95578 (-1.10) & 92641 (0.60) \\ \cline{2-6}
& \ding{55} & sparse & generic mean & 94184 (-0.29) & 92066 (0.94) \\ \cline{2-6}
& \ding{55} & sparse & generic median & 94123 (-0.26) & 92160 (0.88) \\ \cline{2-6}
& \ding{55} & sparse & specific mean & 94083 (-0.24) & 92155 (0.89) \\ \cline{2-6}
& \ding{55} & sparse & specific median & 94494 (-0.47) & 92228 (0.84) \\ \cline{2-6}
& \ding{51} & \ding{55} & \ding{55} & 94509 (-0.48) & 92358 (0.77) \\ \cline{2-6}
& \ding{51} & \ding{55} & generic mean & 94120 (-0.26) & 92547 (0.66) \\ \cline{2-6}
& \ding{51} & \ding{55} & generic median & 93979 (-0.17) & 92485 (0.69) \\ \cline{2-6}
& \ding{51} & \ding{55} & specific mean & 94205 (-0.31) & 92481 (0.70) \\ \cline{2-6}
& \ding{51} & \ding{55} & specific median & 94344 (-0.39) & 92591 (0.63) \\ \cline{2-6}
& \ding{51} & complete & \ding{55} & 95078 (-0.81) & 92399 (0.74) \\ \cline{2-6}
& \ding{51} & complete & generic mean & 94315 (-0.37) & 92166 (0.88) \\ \cline{2-6}
& \ding{51} & complete & generic median & 94591 (-0.53) & 92263 (0.82) \\ \cline{2-6}
& \ding{51} & complete & specific mean & 94525 (-0.49) & 92198 (0.86) \\ \cline{2-6}
& \ding{51} & complete & specific median & 94691 (-0.59) & 92301 (0.80) \\ \cline{2-6}
& \ding{51} & sparse & \ding{55} & 95071 (-0.81) & 92399 (0.74) \\ \cline{2-6}
& \ding{51} & sparse & generic mean & 94315 (-0.37) & 92166 (0.88) \\ \cline{2-6}
& \ding{51} & sparse & generic median & 94591 (-0.53) & 92263 (0.82) \\ \cline{2-6}
& \ding{51} & sparse & specific mean & 94525 (-0.49) & 92198 (0.86) \\ \cline{2-6}
& \ding{51} & sparse & specific median & 94691 (-0.59) & 92301 (0.80) \\ \hline
\multicolumn{4}{c|}{natural order} & 93678 & 93678 \\ \hline
\multicolumn{4}{c|}{``aeiou\ldots''} & 94139 (-0.27) & 93752 (-0.04) \\ \hline
\multicolumn{4}{c|}{histogram differences} & 94531 (-0.50) & 93643 (0.02) \\ \hline
\multicolumn{4}{c|}{number of inversions} & 93732 (-0.03) & 92660 (0.59) \\ \hline
\multicolumn{4}{c|}{number of inversions log} & 94064 (-0.22) & 92823 (0.50) \\ \hline
\end{tabular}
\caption{Simulated compression results for obj1 of the Calgary Corpus using one
order computed with different metrics. In parentheses, the increase in
compression ratio compared to the natural order, in percentage points. All sizes
in bits without overhead, size of the input is 172032 bits.}
\end{table}

\addtocontents{toc}{\setcounter{tocdepth}{-1}}
\chapter{Eidesstattliche Versicherung}

Ich versichere hiermit, dass ich diese Arbeit selbstständig verfasst habe und
keine anderen als die angegebenen Quellen und Hilfsmittel verwendet habe.

\addtocontents{toc}{\setcounter{tocdepth}{3}}

\bibliographystyle{plain}
\bibliography{bib}
\end{document}
