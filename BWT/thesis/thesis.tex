%%This is a very basic article template.
%%There is just one section and two subsections.
\documentclass[a4paper]{scrreprt}

\begin{document}

\chapter{Intro}

-def compression (decompression necessary)
-explain bw-mtf-entropy, mtf can be skipped
-explain simple version of bw used here, no rle
-definitions: context block, symbol (byte in this thesis)

\chapter{Simple Sorting}

\section{Presious Work}

chapin original \cite{chapin1998sort,chapin2001diss}

\section{Intro}

-overwork in context transitions
-metrics to rank transitions, tsp to find a tour
-problem finding atsp heuristic, only one i found only supports ints, give 
	error bounds for the scaling
-chapins metrics work on the bw code
-badness works on the mtf of the bw, with modifications attempts to give the
	number of bits a transition costs:
	-make a list of the best possible alphabet the left side of a transition
		can leave for the right side
	-compare those with the actual codes you get when you mtf encode the left
		and right side together
	-the sum of all the differences is the badness
	-variants:
		-weighting: divide the badness by the number of new symbols on the right
			side, so transitions to a larger context don't get a massive
			penalty
		-new penalty: new symbols in the right side that didn't appear in the
			left side need to receive a penalty instead of assuming the minimum
			possible (mean mtf code for mtf codes of at least the min possible,
			symbol specific)
			prediction for the block that ends up as the first will be way off
		-entropy code len: don't take the plain difference but the difference
			between the number of bits the entropy coder will use. this can
			give you the exact number of bits this transition costs
			approximation with const values for zeros: zeros will all get the
			same (longest possible) code length, other rare symbols may be
			affected
			approximation with curve fitting: zeros will get about the same
			code lengths as the surrounding symbols, but all of the rarer
			symbols (>10?) will have (significantly) longer codes as there are
			more symbols to encode. higher value mtf codes are usually sparse,
			but can't be sure how many zeros will remain after reordering
			curve fitting sparse: assume there will be the same amount of
			zeros after reordering. give zeros the same length as their
			neighbors, even though there can't be a huffman code with those
			code lengths and that many symbols 
	-describe things needed for an ideal badness metric (new penalty,entropy
		len, weighting?), then give approximations
	-another point of non-optimality: which transition to choose as a starting
		point? need to break up the baddest transition
	-need need a new penalty dict that is specific for the symbol being encoded
		i.e. what is the average code for e.g. 'a' given that the minimum
		possible is n


\section{Further Work}

-sorting more useful if using a different list update algo? (with slower
convergence)

\chapter{Exceptions to MTF}

\section{Previous Work}

chapin tried using two list update algos
\cite{chapin2000switching,chapin2001diss}. also assumes that one algo isn't
universally the best

wirth, moffat, altered or no list update \cite{wirth2001ranks}

\bibliographystyle{plain}
\bibliography{bib}
\end{document}
